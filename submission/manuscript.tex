% Options for packages loaded elsewhere
\PassOptionsToPackage{unicode}{hyperref}
\PassOptionsToPackage{hyphens}{url}
%
\documentclass[
  11pt,
]{article}
\usepackage{lmodern}
\usepackage{amssymb,amsmath}
\usepackage{ifxetex,ifluatex}
\ifnum 0\ifxetex 1\fi\ifluatex 1\fi=0 % if pdftex
  \usepackage[T1]{fontenc}
  \usepackage[utf8]{inputenc}
  \usepackage{textcomp} % provide euro and other symbols
\else % if luatex or xetex
  \usepackage{unicode-math}
  \defaultfontfeatures{Scale=MatchLowercase}
  \defaultfontfeatures[\rmfamily]{Ligatures=TeX,Scale=1}
\fi
% Use upquote if available, for straight quotes in verbatim environments
\IfFileExists{upquote.sty}{\usepackage{upquote}}{}
\IfFileExists{microtype.sty}{% use microtype if available
  \usepackage[]{microtype}
  \UseMicrotypeSet[protrusion]{basicmath} % disable protrusion for tt fonts
}{}
\makeatletter
\@ifundefined{KOMAClassName}{% if non-KOMA class
  \IfFileExists{parskip.sty}{%
    \usepackage{parskip}
  }{% else
    \setlength{\parindent}{0pt}
    \setlength{\parskip}{6pt plus 2pt minus 1pt}}
}{% if KOMA class
  \KOMAoptions{parskip=half}}
\makeatother
\usepackage{xcolor}
\IfFileExists{xurl.sty}{\usepackage{xurl}}{} % add URL line breaks if available
\IfFileExists{bookmark.sty}{\usepackage{bookmark}}{\usepackage{hyperref}}
\hypersetup{
  hidelinks,
  pdfcreator={LaTeX via pandoc}}
\urlstyle{same} % disable monospaced font for URLs
\usepackage[margin=1.0in]{geometry}
\usepackage{graphicx,grffile}
\makeatletter
\def\maxwidth{\ifdim\Gin@nat@width>\linewidth\linewidth\else\Gin@nat@width\fi}
\def\maxheight{\ifdim\Gin@nat@height>\textheight\textheight\else\Gin@nat@height\fi}
\makeatother
% Scale images if necessary, so that they will not overflow the page
% margins by default, and it is still possible to overwrite the defaults
% using explicit options in \includegraphics[width, height, ...]{}
\setkeys{Gin}{width=\maxwidth,height=\maxheight,keepaspectratio}
% Set default figure placement to htbp
\makeatletter
\def\fps@figure{htbp}
\makeatother
\setlength{\emergencystretch}{3em} % prevent overfull lines
\providecommand{\tightlist}{%
  \setlength{\itemsep}{0pt}\setlength{\parskip}{0pt}}
\setcounter{secnumdepth}{-\maxdimen} % remove section numbering
\usepackage{helvet} % Helvetica font
\renewcommand*\familydefault{\sfdefault} % Use the sans serif version of the font
\usepackage[T1]{fontenc}

\usepackage[none]{hyphenat}

\usepackage{setspace}
\doublespacing
\setlength{\parskip}{1em}

\usepackage{lineno}

\usepackage{pdfpages}

\author{}
\date{\vspace{-2.5em}}

\begin{document}

\vspace{35mm}

\hypertarget{an-osmotic-laxative-renders-mice-susceptible-to-prolonged-clostridioides-difficile-colonization-and-hinders-clearance}{%
\section{\texorpdfstring{An osmotic laxative renders mice susceptible to
prolonged \emph{Clostridioides difficile} colonization and hinders
clearance}{An osmotic laxative renders mice susceptible to prolonged Clostridioides difficile colonization and hinders clearance}}\label{an-osmotic-laxative-renders-mice-susceptible-to-prolonged-clostridioides-difficile-colonization-and-hinders-clearance}}

\vspace{35mm}

Sarah Tomkovich\^{}1, Ana Taylor, Jacob Kingg, Joanna Colovas, Lucas
Bishop, Kathryn McBride, Sonya Royzenblat, Nicholas A. Lesniak, Ingrid
L. Bergin\^{}2, Patrick D. Schloss\textsuperscript{1\(\dagger\)}

\vspace{40mm}

\(\dagger\) To whom correspondence should be addressed:
\href{mailto:pschloss@umich.edu}{\nolinkurl{pschloss@umich.edu}}

1. Department of Microbiology and Immunology, University of Michigan,
Ann Arbor, MI, USA

2. The Unit for Laboratory Animal Medicine, University off Michigan, Ann
Arbor, MI, USA

\newpage
\linenumbers

\hypertarget{abstract}{%
\subsection{Abstract}\label{abstract}}

(Modify depending on target journal) Antibiotics are a major risk factor
for \emph{Clostridioides difficile} infections (CDIs) because of their
impact on the intestinal microbiome. However, non-antibiotic medications
such as the ubiquitous osmotic laxative polyethylene glycol (PEG) 3350,
also alter the microbiota, but whether PEG impacts CDI susceptibility
and clearance is unclear. To examine how PEG impacts susceptibility, we
treated C57Bl/6 mice with 5-day and 1-day doses of 15\% PEG in the
drinking water and then challenged the mice with \emph{C. difficile} 630
spores. We used clindamycin-treated mice as a control because they
consistently clear \emph{C. difficile} within 10 days post-infection
(dpi). To examine how PEG treatment impacts clearance, we administered
PEG for 1 day to clindamycin-treated, \emph{C. difficile}-challenged
mice either immediately following challenge or 3 dpi. We collected
longitudinal stool samples to examine \emph{C. difficile} levels in the
stool via anaerobic culture and profiled the microbiota by 16S rRNA
sequencing. PEG treatment alone was sufficient to render mice
susceptible to CDI and 5-day PEG-treated mice remain colonized for up to
30 dpi. Additionally, 5-day PEG treated mice remained susceptible to CDI
10-days post treatment. In contrast, 1-day PEG treated mice were
transiently colonized, clearing \emph{C. difficile} within 7 dpi.
Although 5-day PEG-treated mice exhibited prolonged \emph{C. difficile}
colonization, we saw no difference in histological inflammation between
PEG- and clindamycin-treated mice. Additionally, administering PEG to
mice after \emph{C. difficile} challenge prolonged colonization up to 30
dpi in mice that received PEG immediately after challenge and 15 dpi in
mice that received PEG 3 dpi. When we examined microbiota composition
across our different treatment groups, we found decreased richness in
the PEG-treated mice that exhibited prolonged \emph{C. difficile}
colonization. Importantly, there were increased Bacteroides and
Enterobacteriaceae and decreased Lachnospiraceae and Oscillibacter in
most of the PEG-treated mice with prolonged \emph{C. difficile}
colonization. Our findings suggest the osmotic laxative PEG 3350 alters
the mouse microbiota and disrupts colonization resistance to \emph{C.
difficile}, as well as clearance in mice with a CDI. Considering that
most hospitals recommend not performing \emph{C. difficile} testing on
patients taking laxatives and laxatives are used when administering
fecal microbiota transplants via colonoscopy to patients with recurrent
CDIs, further studies are needed to evaluate if laxatives impact human
microbiota colonization resistance.

\newpage

\hypertarget{introduction}{%
\subsection{Introduction}\label{introduction}}

\hypertarget{results-and-discussion}{%
\subsection{Results and Discussion}\label{results-and-discussion}}

\hypertarget{conclusions}{%
\subsection{Conclusions}\label{conclusions}}

\hypertarget{acknowledgements}{%
\subsection{Acknowledgements}\label{acknowledgements}}

We thank members of the Schloss lab for feedback on planning the
experiments and data presentation. We also thank Andrew Henry for help
with media preparation and bacterial culture. We also thank the Unit for
Laboratory Animal Medicine at the University of Michigan for maintaining
our mouse colony and providing the institutional support for our mouse
experiments. Finally, we thank Kwi Kim, Austin Campbell, and Kimberly
Vendrov for their help in maintaining the Schloss lab's anaerobic
chamber. This work was supported by the National Institutes of Health
(U01AI124255). ST was supported by the Michigan Institute for Clincial
and Health Research Postdoctoral Translation Scholars Program
(UL1TR002240 from the National Center for Advancing Translational
Sciences).

\hypertarget{materials-and-methods}{%
\subsection{Materials and Methods}\label{materials-and-methods}}

\newpage

\includegraphics{figure_1.pdf} \textbf{Figure 1. Chronic PEG treatment
prolongs susceptibility and mice become persistently colonized with
\emph{C. difficile}.} A. Setup of the experimental timeline for subset
of experiments with 5-day PEG treated mice. B. \emph{C. difficile}
CFU/gram stool measured over time (N =
4-\texttt{(insert\ variable\ name)} mice per timepoint) via serial
dilutions. The black line represents the limit of detection for the
first serial dilution. CFU quantification data was not available for
each mouse due to stool sampling difficulties (particularly the day the
mcie came off of the PEG treatment) or early deaths. Lines represent the
median for each source and circles represent individual mouse samples.

\newpage

\hypertarget{references}{%
\subsection{References}\label{references}}

\end{document}
