% Options for packages loaded elsewhere
\PassOptionsToPackage{unicode}{hyperref}
\PassOptionsToPackage{hyphens}{url}
%
\documentclass[
  11pt,
]{article}
\usepackage{lmodern}
\usepackage{amssymb,amsmath}
\usepackage{ifxetex,ifluatex}
\ifnum 0\ifxetex 1\fi\ifluatex 1\fi=0 % if pdftex
  \usepackage[T1]{fontenc}
  \usepackage[utf8]{inputenc}
  \usepackage{textcomp} % provide euro and other symbols
\else % if luatex or xetex
  \usepackage{unicode-math}
  \defaultfontfeatures{Scale=MatchLowercase}
  \defaultfontfeatures[\rmfamily]{Ligatures=TeX,Scale=1}
\fi
% Use upquote if available, for straight quotes in verbatim environments
\IfFileExists{upquote.sty}{\usepackage{upquote}}{}
\IfFileExists{microtype.sty}{% use microtype if available
  \usepackage[]{microtype}
  \UseMicrotypeSet[protrusion]{basicmath} % disable protrusion for tt fonts
}{}
\makeatletter
\@ifundefined{KOMAClassName}{% if non-KOMA class
  \IfFileExists{parskip.sty}{%
    \usepackage{parskip}
  }{% else
    \setlength{\parindent}{0pt}
    \setlength{\parskip}{6pt plus 2pt minus 1pt}}
}{% if KOMA class
  \KOMAoptions{parskip=half}}
\makeatother
\usepackage{xcolor}
\IfFileExists{xurl.sty}{\usepackage{xurl}}{} % add URL line breaks if available
\IfFileExists{bookmark.sty}{\usepackage{bookmark}}{\usepackage{hyperref}}
\hypersetup{
  hidelinks,
  pdfcreator={LaTeX via pandoc}}
\urlstyle{same} % disable monospaced font for URLs
\usepackage[margin=1.0in]{geometry}
\usepackage{graphicx,grffile}
\makeatletter
\def\maxwidth{\ifdim\Gin@nat@width>\linewidth\linewidth\else\Gin@nat@width\fi}
\def\maxheight{\ifdim\Gin@nat@height>\textheight\textheight\else\Gin@nat@height\fi}
\makeatother
% Scale images if necessary, so that they will not overflow the page
% margins by default, and it is still possible to overwrite the defaults
% using explicit options in \includegraphics[width, height, ...]{}
\setkeys{Gin}{width=\maxwidth,height=\maxheight,keepaspectratio}
% Set default figure placement to htbp
\makeatletter
\def\fps@figure{htbp}
\makeatother
\setlength{\emergencystretch}{3em} % prevent overfull lines
\providecommand{\tightlist}{%
  \setlength{\itemsep}{0pt}\setlength{\parskip}{0pt}}
\setcounter{secnumdepth}{-\maxdimen} % remove section numbering
\usepackage{helvet} % Helvetica font
\renewcommand*\familydefault{\sfdefault} % Use the sans serif version of the font
\usepackage[T1]{fontenc}

\usepackage[none]{hyphenat}

\usepackage{setspace}
\doublespacing
\setlength{\parskip}{1em}

\usepackage{lineno}

\usepackage{pdfpages}

\author{}
\date{\vspace{-2.5em}}

\begin{document}

\vspace{35mm}

\hypertarget{an-osmotic-laxative-renders-mice-susceptible-to-prolonged-clostridioides-difficile-colonization-and-hinders-clearance}{%
\section{\texorpdfstring{An osmotic laxative renders mice susceptible to
prolonged \emph{Clostridioides difficile} colonization and hinders
clearance}{An osmotic laxative renders mice susceptible to prolonged Clostridioides difficile colonization and hinders clearance}}\label{an-osmotic-laxative-renders-mice-susceptible-to-prolonged-clostridioides-difficile-colonization-and-hinders-clearance}}

\vspace{35mm}

Sarah Tomkovich\textsuperscript{1}, Ana Taylor\textsuperscript{1}, Jacob
King\textsuperscript{1}, Joanna Colovas\textsuperscript{1}, Lucas
Bishop\textsuperscript{1}, Kathryn McBride\textsuperscript{1}, Sonya
Royzenblat\textsuperscript{1}, Nicholas A. Lesniak\textsuperscript{1},
Ingrid L. Bergin\textsuperscript{2}, Patrick D.
Schloss\textsuperscript{1\(\dagger\)}

\vspace{40mm}

\(\dagger\) To whom correspondence should be addressed:
\href{mailto:pschloss@umich.edu}{\nolinkurl{pschloss@umich.edu}}

1. Department of Microbiology and Immunology, University of Michigan,
Ann Arbor, MI, USA

2. The Unit for Laboratory Animal Medicine, University of Michigan, Ann
Arbor, MI, USA

\newpage
\linenumbers

\hypertarget{abstract}{%
\subsection{Abstract}\label{abstract}}

Antibiotics are a major risk factor for \emph{Clostridioides difficile}
infections (CDIs) because of their impact on the microbiota. However,
non-antibiotic medications such as the ubiquitous osmotic laxative
polyethylene glycol (PEG) 3350, also alter the microbiota, but whether
PEG impacts CDI susceptibility and clearance is unclear. To examine how
PEG impacts susceptibility, we treated C57Bl/6 mice with 5-day and 1-day
doses of 15\% PEG in the drinking water and then challenged the mice
with \emph{C. difficile} 630. We used clindamycin-treated mice as a
control because they consistently clear \emph{C. difficile} within 10
days post-challenge (dpc). PEG treatment alone was sufficient to render
mice susceptible and 5-day PEG-treated mice remain colonized for up to
30 dpc. In contrast, 1-day PEG treated mice were transiently colonized,
clearing \emph{C. difficile} within 7 dpc. To examine how PEG treatment
impacts clearance, we administered a 1-day PEG treatment to
clindamycin-treated, \emph{C. difficile}-challenged mice. Administering
PEG to mice after \emph{C. difficile} challenge prolonged colonization
up to 30 dpc. When we trained a random forest model with community data
from 5 dpc, we were able to predict which mice would exhibit prolonged
colonization (AUROC = 0.90). Five of the top ten bacterial features
important for predicting prolonged colonization had high relative
abundances in the community. Examining the dynamics of these bacterial
during the post-challenge period revealed patterns in the relative
abundances of \emph{Bacteroides}, \emph{Enterobacteriaceae},
\emph{Porphyromonadaceae}, \emph{Lachnospiraceae}, and
\emph{Akkermansia} that were associated with prolonged \emph{C.
difficile} colonization in PEG-treated mice.

\hypertarget{importance}{%
\subsection{Importance}\label{importance}}

Diarrheal samples induced by medications such as laxatives are typically
rejected for \emph{Clostridiodes difficile} testing. However, there are
some microbiota similarities between diarrheal and \emph{C. difficile}
infection (CDI) communities such as lower diversity compared to healthy
controls, which lead us to hypothesize that diarrhea may be an indicator
of \emph{C. difficile} risk. We explored how osmotic laxatives disrupt
the microbiota's colonization resistance to \emph{C. difficile} by
administering a laxative to mice either before or after \emph{C.
difficile} challenge Our findings suggest the osmotic laxative disrupts
colonization resistance to \emph{C. difficile}, as well as clearance in
mice already colonized with \emph{C. difficile}. Considering that most
hospitals recommend not performing \emph{C. difficile} testing on
patients taking laxatives and laxatives are used when administering
fecal microbiota transplants via colonoscopy to patients with recurrent
CDIs, further studies are needed to evaluate if laxatives impact
microbiota colonization resistance in humans.

\newpage

\hypertarget{introduction}{%
\subsection{Introduction}\label{introduction}}

Antibiotics are a major risk factor for \emph{Clostridioides difficile}
infections (CDIs) because they disrupt microbiota colonization
resistance (1). However, antibiotics are not the only types of
medications that disrupt the microbiota (2--4). Although, other
medications (proton pump inhibitors, osmotic laxatives, antimotility
agents, and opioids) have been implicated as risk or protective factors
for CDIs through epidemiological studies, whether the association is due
to their impact on the microbiota is still unclear (5--9).

Many of the non-antibiotic medications associated with CDIs are known to
modulate gastrointestinal motility leading to either increased or
decreased colonic transit time, which in turn also strongly impacts
microbiota composition and function (10, 11). Stool consistency often
serves as an approximation of intestinal motility (10). Our group has
shown that when \emph{C. difficile} negative controls are separated into
two groups based on stool consistency, there are shared microbiota
features such as lower alpha diversity in samples from CDI patients and
control patients with diarrhea compared to control samples that were
\emph{C. difficile} negative with non-diarrheal consistency (12). These
results led to a hypothesis that bacterial communities from patients
experiencing diarrhea are susceptible to developing CDIs.

Osmotic laxatives can lead to diarrhea depending on the administered
dose and temporarily disrupt the human intestinal microbiota (13). The
ubiquitous osmotic laxative, polyethylene glycol (PEG) 3350 is found in
Miralax, Nulytely, and Golytely and is also commonly used as bowel
preparation for colonoscopies. Interestingly, previous studies have
shown that treating mice with PEG alone altered microbiota composition,
reduced acetate and butyrate production, altered the mucus barrier, and
rendered the mice susceptible to \emph{C. difficile} colonization
(14--17). The mucus barrier is thought to mediate protection from CDIs
by protecting intestinal epithelial cells from the toxins produced by
\emph{C. difficile} (18, 19). Whether laxative administration results in
more severe CDIs in mice and how long mice remain colonized with
\emph{C. difficile} after challenge is unclear.

Beyond susceptibility, PEG is also relevant in the context of treating
recurrent CDIs via fecal microbiota transplant (FMT) where a healthy
microbiota is administered to the patient to restore colonization
resistance. For FMTs that are delivered via colonoscopy, patients
typically undergo bowel preparation by taking an osmotic laxative prior
to the procedure. Many of the FMT studies to date rationalize the use of
laxatives (20--22) based on a 1996 case study with 2 pediatric patients
where the authors suggested in the discussion that the laxative may help
flush \emph{C. difficile} spores and toxins from the intestine (23).

Our group has used C57BL/6 mice to characterize how antibiotics disrupt
the microbiota and influence \emph{C. difficile} susceptibility and
clearance (24--26). Although, two groups have now shown PEG treatment
alone renders mice susceptible to \emph{C. difficile} (15, 17), these
studies have raised additional questions regarding the dynamics and
severity of infection as well as the role of laxative treatment in
\emph{C. difficile} clearance. Addressing these questions will better
inform how we think about laxatives in the context of CDIs. Here, we
characterized how long PEG-treated mice remain susceptible, whether PEG
treatment results in more sustained \emph{C. difficile} colonization and
severe CDI than mice treated with clindamycin, and if PEG treatment
after challenge can promote \emph{C. difficile} clearance.

\hypertarget{results}{%
\subsection{Results}\label{results}}

\textbf{5-day laxative treatment led to prolonged \emph{C. difficile}
colonization in mice.} Building off of previous work that showed
treating mice with the osmotic laxative, PEG 3350, rendered mice
susceptible to \emph{C. difficile} colonization (15, 17), we decided to
test how long \emph{C. difficile} colonization is sustained and how long
PEG-treated mice remain susceptible to \emph{C. difficile}. We compared
three groups of mice treated with PEG 3350 to one group of mice treated
with our standard 10 mg/kg clindamycin treatment, which temporarily
renders the mice susceptible to \emph{C. difficile} colonization, with
mice typically clearing \emph{C. difficile} within 10 days
post-challenge (9, 26). All three groups of PEG-treated mice were
administered a 15\% PEG solution in the drinking water for 5-days: 1.
received no additional treatment, 2. was also treated with clindamycin,
and 3. was allowed to recover for 10 days prior to challenge (Fig. 1A).
PEG treatment resulted in weight loss in all 3 groups of PEG-treated
mice relative to their baseline weights, with the greatest change in
weight observed on the fifth day of PEG treatment. The mice recovered
most of the lost weight by five days after treatment (Fig. 1B). After
either the PEG, clindamycin, or PEG and clindamycin treatment all mice
were challenged with 10\textsuperscript{5} \emph{C. difficile} 630
spores (Fig. 1A). All treatments rendered mice susceptible to \emph{C.
difficile} colonization. However, PEG-treated mice remained colonized
with \emph{C. difficile} at a high level through thirty days
post-challenge (Fig. 1C). In contrast, the clindamycin-treated mice
cleared \emph{C. difficile} within ten days post-challenge (Fig. 1C).
Therefore, PEG treatment led to sustained colonization in contrast to
clindamycin mice that naturally cleared \emph{C. difficile} within ten
days post-challenge.

Notably, we also found PEG-treated mice were still susceptible to
\emph{C. difficile} colonization after a 10-day recovery period,
although \emph{C. difficile} was not detectable in most of the group in
the initial five days post-challenge (Fig. 1C, S1A). One mouse was found
dead on the 6th day post-challenge, presumably due to \emph{C.
difficile} as the bacteria became detectable in stool samples from that
mouse on the 4th day post-challenge (Fig. S1A, mouse 10). From 8 days
post-challenge onward, the density of \emph{C. difficile} stabilized in
the 10-day recovery group and remained high through 20-30 days
post-challenge (Fig. 1C). Thus, osmotic laxative treatment alone was
sufficient to render mice susceptible to prolonged \emph{C. difficile}
colonization and PEG-treated mice remained susceptible through ten days
post-treatment.

\textbf{5-day laxative treatment differentially disrupted the fecal
microbiota compared to clindamycin treatment.} Since osmotic laxatives
and clindamycin have previously been shown to disrupt the murine
microbiota (14--17), we hypothesized the different \emph{C. difficile}
colonization dynamics between mice treated with the osmotic laxative or
clindamycin were due to the two drugs having differential effects on the
microbiota. We profiled the stool microbiota over time by sequencing the
V4 region of the 16S rRNA gene to compare changes across treatment
groups. We found time and treatment group explained half of the observed
variation between fecal communities with most of the remaining variation
explained by interactions between treatment group and other experimental
variables including time, cage effects, and sequencing preparation plate
(PERMANOVA combined R\textsuperscript{2} = 0.95, \emph{P} \textless{}
0.001, Fig. 2A, Data Set S1, Sheet X). Cage effects refer to the
well-documented phenomenon that mice housed in the same cages have
similar microbial communities due to coprophagy (27). We tried to
minimize the impact of cage effects on our experiment by breaking up
littermates when assigning mice to treatment groups. Importantly,
although we conducted a total of 5 separate experiments, the experiment
number and its interaction with treatment group did no significantly
explain the observed variation in fecal communities (Data Set S1, Sheet
X). None of the treatment groups recovered to their baseline community
structure either 10 or 30 days post-challenge suggesting other community
features besides recovery to baseline were responsible for the prolonged
\emph{C. difficile} colonization in PEG-treated mice (Fig. 2B).

Since none of the communities completely recovered in the follow-up
period after treatments, we next profiled community diversity and
composition. We examined the alpha diversity dynamics by calculating the
communities' Shannon diversity. Although both clindamycin and PEG
treatments decreased diversity, Shannon diversity was lower in the
groups of mice that received PEG treatment compared to those that
received clindamycin through thirty days post-challenge (Fig. 2C). We
next examined the bacterial genera that shifted after PEG treatment by
comparing the baseline samples of mice treated with only PEG to samples
from the same mice one day post-PEG-treatment. We found 18 genera that
were altered by PEG treatment (Data Set S1, Sheet X). The majority of
the bacterial relative abundances decreased after the PEG treatment, but
the relative abundance among members of the \emph{Enterobacteriaceae}
and \emph{Bacteroides} increased. The increase in \emph{Bacteroides}
relative abundance was unique to PEG treated mice, as the
\emph{Bacteroides} relative abundance actually decreased in clindamycin
treated mice (Fig. 2D). Finally, we identified the genera whose relative
abundance differed across treatment groups over multiple time points. Of
the 33 genera that were different between treatment groups, 24 genera
were different over multiple time points (Fig. 2E, Data Set S1, Sheet
X). Thus, PEG had a significant impact on the fecal microbiota that was
maintained over time and was distinct from clindamycin treatment.

Interestingly, \emph{C. difficile} was not immediately detectable in the
stools of the PEG-treated mice that were allowed to recover for 10 days
prior to challenge. We decided to examine if there were genera that
changed during the post-challenge period when the group median \emph{C.
difficile} shifted from undetectable at 1 day post-challenge to
detectable in the stool samples with the density stabilizing around 8
days post-challenge (Fig. S1A). We found no bacteria with relative
abundances that were significantly different over the two time points
after multiple hypothesis correction (Data Set S1, Sheet X). However,
there was also wide variation between individual mice regarding when
\emph{C. difficile} became detectable (Fig. S1A) as well as the relative
abundances of bacterial genera in the communities (Fig. S1B). For
example, two mice had a high relative abundance of
\emph{Enterobacteriaceae} throughout the post-challenge period and this
corresponded to mouse 10, which died on the sixth day post-challenge and
mouse 11, where \emph{C. difficile} was present at a high density from
the 4th day post-challenge onward (Fig. S1B). While we did not identify
a clear signal to explain the delayed appearance of \emph{C. difficile}
in the 5-day PEG mice that were allowed to recover for 10 days prior to
challenge, the delay is striking and could reflect changes in microbial
activity or metabolites that were not examined in this study.

\textbf{5-day laxative treatment did not promote more severe CDIs
despite altering the mucosal microbiota.} Given the findings from a
previous study that demonstrated PEG treatment disrupts the mucus layer
and alters the immune response in mice (16), we decided to examine the
impact of PEG treatment on the mucosal microbiota and CDI severity. To
evaluate the mucosal microbiota, we sequenced snips of tissue collected
from the cecum, proximal colon, and distal colon. Similar to what was
observed with the stool samples, alpha diversity was lower in the
PEG-treated mice compared to clindamycin treated mice (Fig. 3A). Alpha
diversity continued to increase over time based on the tissues from
PEG-treated mice collected at 20 and 30 days post-challenge (Fig. 3A,
Data Set S1, Sheet X). Group, time point, and their interactions with
other variables (cage, experiment number, and sample type) explained the
majority of the variation observed in mucosal communities (PERMANOVA
combined R\textsuperscript{2} = 0.83, \emph{P} \textless{} 0.05, Fig.
3B, Data Set S1, Sheet X). We saw the greatest difference in the
relative abundance of the mucosal microbiota between treatment groups
(clindamycin, 5-day PEG, and 5-day PEG plus clindamycin) at 6 days
post-challenge with 10 genera that were significantly different
(\emph{P} \textless{} 0.05) in all three of the tissue types we
collected (cecum, proximal colon, and distal colon; Fig. S2A, Data Set
S1, Sheet X). Interestingly, \emph{Peptostreptococcaceae} (the family
with a sequence that matches \emph{C. difficile}) was one of the genera
that had a significant difference in relative abundance between
treatment groups at 6 days post-challenge. This population was primarily
only present in the 5-day PEG treatment group of mice and decreased in
the proximal and distal colon tissues over time (Fig. S2B). By 30 days
post-challenge, only the relative abundances of \emph{Bacteroides},
\emph{Clostridiales}, \emph{Firmicutes}, and \emph{Ruminococcaceae} were
different between treatment groups and only in the cecum tissues (Fig.
3C, Fig. 2E, Data Set S1, Sheet X). Thus, PEG treatment had a
significant impact on the mucosal microbiota and we detected \emph{C.
difficile} sequences in the cecum, proximal colon, and distal colon
tissue communities.

Because there were differences in the mucosal microbiota including
detectable \emph{C. difficile} sequences in tissues from PEG-treated
mice relative to mice treated with clindamycin, we next examined the
severity of \emph{C. difficile} challenge by evaluating cecum and colon
H\&E stained histopathology (28). However, we found there was no
difference in cecum and colon scores between clindamycin and PEG-treated
mice that were challenged with \emph{C. difficile} at 4 days
post-challenge (Fig. 3D), the time point typically examined in \emph{C.
difficile} 630 challenged mice (29). We also looked at 6 days
post-challenge because that was when there was a large difference in
\emph{C. difficile} density between PEG- and clindamycin-treated mice
(Fig. 1C). Although, there was a slight difference in the colon between
PEG and clindamycin-treated mice, there was not a signifant difference
in the cecum and the overall score was relatively low (1.5-2.5 out of
12, Fig. 3E). Therefore, although PEG treatment had a disruptive effect
on the mucosal microbiota, the impact of \emph{C. difficile} 630
challenge on the cecum and colon was similar between PEG and clindamycin
treated mice.

\textbf{\emph{C. difficile} challenge did not have a synergistic
disruptive effect on the microbiota of PEG-treated mice.} Because
\emph{C. difficile} itself can have an impact on the microbiota (30), we
also sequenced the tissue and stools of mock-challenged clindamycin and
5-day PEG treated mice. Examining the stools of the mock-challenged mice
revealed similar bacterial disruptions as the \emph{C. difficile}
challenged mice (Fig. S3A-C). Similarly, there was no difference between
the tissues of mock and \emph{C. difficile} challenged mice (Fig.
S3D-F). Thus, most of the microbiota alterations we observed in the
PEG-treated mice were a result of the laxative and not an interaction
between the laxative and \emph{C. difficile}.

\textbf{1-day laxative treatment resulted in transient \emph{C.
difficile} colonization and minor microbiota disruption.} Next, we
examined how a shorter osmotic laxative perturbation would impact the
microbiome and susceptibility to \emph{C. difficile}. We administered
either a 1-day PEG treatment, a 1-day PEG treatment with a 1-day
recovery period, or clindamycin to mice before challenging them with
\emph{C. difficile} (Fig. 3A). In contrast to the 5-day PEG treated
mice, the 1-day PEG treated mice were only transiently colonized and
cleared \emph{C. difficile} by 7 days post-challenge (Fig. 3B). The
stool communities of PEG-treated mice were also only transiently
disrupted, with Shannon diversity recovering by 7 days post-challenge
(Fig. 3C-D). We found the relative abundances of 14 genera were impacted
by treatment, but recovered close to baseline levels by 7 days
post-challenge including \emph{Enterobacteriaceae},
\emph{Clostridiales}, \emph{Porpyromonadaceae}, and
\emph{Ruminococcaceae} (Fig. 3E, Data Set S1, Sheet X). These findings
suggest the duration of the PEG treatment was relevant, with shorter
treatments resulting in a transient loss of \emph{C. difficile}
colonization resistance.

\textbf{Post-challenge laxative treatment disrupted clearance in
clindamycin-treated mice regardless of whether an FMT was also
administered.} Since a 1-day PEG treatment resulted in a more mild
microbiota perturbation, we decided to use the 1-day treatment to
examine the hypothesis that PEG helps to flush \emph{C. difficile}
spores from the intestine. To examine the impact of PEG treatment on
\emph{C. difficile} clearance, we treated 4 groups of mice with
clindamycin and then challenged all mice with \emph{C. difficile} before
administering the following treatments: no additional treatment, 1-day
PEG immediately after challenge, and 1-day PEG treatment 3 days after
challenge followed by either administration of an FMT or PBS solution by
oral gavage (Fig. 5A). Contrary to our hypothesis, all groups of mice
that received PEG exhibited prolonged \emph{C. difficile} colonization
(Fig. 5B).

We were also interested in exploring whether PEG might help with
engraftment in the context of FMTs. The FMT appeared to partially
restore Shannon diversity but not richness (Fig. 5C-D). Similarly, we
saw some overlap between the communities of mice that received FMT and
the mice treated with only clindamycin after 5 days post-challenge (Fig.
6A). The increase in Shannon diversity suggests that the FMT did have an
impact on the microbiota, despite seeing prolonged \emph{C. difficile}
colonization in the FMT treated mice. However, only the relative
abundances of \emph{Bacteroidales} and \emph{Porphyromonadaceae}
consistently differed between the mice received either an FMT or PBS
gavage (Fig. 6B), suggesting the FMT only restored a couple of genera.
Overall, we found the relative abundances of 24 genera were different
between treatment groups over multiple timepoints. For example, the
relative abundance of \emph{Akkermansia} was increased and the relative
abundances of \emph{Ruminococcaceae}, \emph{Clostridiales},
\emph{Lachnospiraceae}, and \emph{Oscillibacter} were decreased in mice
that received PEG after \emph{C. difficile} challenge relative to
clindamycin treated mice (Fig. 6C). In sum, administering PEG actually
prolonged \emph{C. difficile} colonization, including in mice that
received an FMT, which only restored 2 bacterial genera.

\textbf{Five-day post-challenge community data can predict which mice
that will have prolonged \emph{C. difficile} colonization.} After
identifying bacteria associated with the 5-day, 1-day and post-challenge
1-day PEG treatments, we decided to examine the bacteria that influenced
prolonged \emph{C. difficile} colonization. We trained 3 types of
machine learning models (random forest, logistic regression, and support
vector machine) with bacterial community data from 5 days post-challenge
to predict whether the mice were still colonized with \emph{C.
difficile} 10 days post-challenge. We chose 5 days post-challenge
because that was the earliest time point where we would see a treatment
effect in the post-challenge 1-day PEG experiments. The random forest
model had the highest performance (median AUROC = 0.90, Data Set S1,
Sheet X), so we next performed permutation importance to examine the
bacteria that were the top contributors to the random forest model
predicting prolonged \emph{C. difficile }colonization. We selected the
top 10 bacteria contributing to our model's performance (Fig. 7A) and
examined their relative abundance at 5 days post-challenge, the time
point used to predict \emph{C. difficile} colonization status on day 10
(Fig. 7B). Next, we focused on the 5 genera that had a greater than 1 \%
relative abundance in either the cleared or colonized mice and examined
how the bacteria changed over time. We found \emph{Enterobacteriaceae}
and \emph{Bacteroides} tended to have a higher relative abundance, the
relative abundance of \emph{Akkermansia} was initially decreased and
then increased, and \emph{Porphyromonadaceae} and \emph{Lachnospiraceae}
had a lower relative abundance in the mice with prolonged colonization
compared to the mice that cleared \emph{C. difficile} (Fig. 7C).
Together these results suggest a combination of low and high abundance
bacterial genera influence the prolonged colonization observed in 5-day
PEG and post-challenge 1-day PEG treated mice.

\hypertarget{discussion}{%
\subsection{Discussion}\label{discussion}}

While the disruptive effect of antibiotics on \emph{C. difficile}
colonization resistance is well established, the extent to which other
drugs such as laxatives disrupt colonization resistance was unclear. By
following osmotic laxative treated mice over time, we found 5-day PEG
treatment before challenge resulted in prolonged \emph{C. difficile}
colonization, while a 1-day PEG treatment resulted in transient
colonization. The differences in \emph{C. difficile} colonization
dynamics between the 5- and 1-day PEG treated mice were associated with
differences in how much the treatments disrupted the microbiota.
Additionally, the intestinal communities of 5-day PEG treated mice had
not regained colonization resistance after a 10-day recovery period.
Although in contrast to the other 5-day PEG treatment groups, \emph{C.
difficile} was not immediately detectable in the stools of most of the
mice in the 10-day recovery group. We also examined the impact of PEG
treatment after \emph{C. difficile} challenge and in opposition to the
hypothesis suggested by the literature, found that PEG treatment
prolonged colonization relative to mice that only recieved clindamycin
treatment. We identified patterns in the relative abundances of
\emph{Bacteroides}, \emph{Enterobacteriaceae}, \emph{Akkermansia},
\emph{Porphyromonadaceae}, and \emph{Lachnospiraceae} that were
associated with prolonged \emph{C. difficile} colonization (Fig. 8).
Overall, our results demonstrated that osmotic laxative treatment alone
rendered mice susceptible to \emph{C. difficile} colonization and the
duration of colonization depended on the length of PEG treatment and
whether treatment was administered before or after challenge.

In addition to altering composition, laxative treatment may alter
microbiota-produced metabolites. A previous study demonstrated that a
5-day treatment of 10\% PEG depleted acetate and butyrate and increased
succinate compared to untreated mice (15). While we did not perform
metabolomic analysis, we did see bacteria known to produce benefical
metabolites were depleted in mice that cleared \emph{C. difficile}
compared to mice with prolonged colonization. For example,
\emph{Oscillibacter valericigenes} can produce the SCFA valerate (31),
and separate studies demonstrated valerate is depleted after clindamycin
treatment and inhibited \emph{C. difficile} growth based on \emph{in
vitro} data and an experiment with C57BL/6 mice (32, 33). Similarly,
\emph{Acetatifactor} can produce acetate and butyrate (34), SCFAs that
are decreased in mice with prolonged \emph{C. difficile} infection after
antibiotic treatment (35). Thus protective bacteria and their
metabolites could be depleted by osmotive laxative treatment depending
on the timing and duration of treatment.

One possible explanation for the prolonged \emph{C. difficile}
colonization in 5-day PEG treated mice, might be due to the bacteria's
persistence in the mucosal compartment. In fact, it has been
hypothesized that \emph{C. difficile} biofilms may serve as reservoirs
for recurrent infections (36) and \emph{C. difficile} biofilms were
recently identified in patients and aggregates with \emph{Fusobacterium
nucleatum}, another species capable of forming biofilms (37). There was
an interesting pattern of increased \emph{Enterobacteriaceae},
\emph{Bacteroides}, and \emph{C. difficile} in both the stool and
mucosal communities of PEG-treated mice suggesting a potential synergy.
\emph{Bacteroides} has the potential to degrade mucus and the osmotic
laxative may have allowed \emph{Bacteroides} to colonize the mucosal
niche by degrading mucin glycans with glycosyl hydrolases that are
absent in \emph{C. difficile} (38). \emph{Bacteroides} persistent in the
mucosal tissue might also have helped \emph{Enterobacteriaceae} to
colonize the region, as a synergy between mucus-degrading \emph{B.
fragilis} and \emph{E. coli} has previously been described (39). A
separate study demonstrated \emph{C. difficile} was present in the outer
mucus layer and associated with \emph{Enterobacteriaceae} and
\emph{Bacteroidaceae} using fluorescent in situ hybridization (FISH)
staining (40). However, \emph{B. fragilis} prevented CDI morbidity in a
mouse model and inhibited \emph{C. difficile} adherence \emph{in vitro}
(41). In coculture experiments \emph{B. longum} decreased \emph{C.
difficile} biofilm formation while \emph{B. thetaiotamicron} enhanced
biofilm formation (42). Therefore, whether \emph{Bacterodes} is
detrimental or beneficial in the context of \emph{C. difficile}
infection or colonization is still unclear.

\emph{Akkermansia} is also a mucin degrader with potentially beneficial
or detrimental roles depending on context in other diseases (43, 44). In
our study the relative abundance of \emph{Akkermansia} shifted over time
between groups. In the stool it was initially increased in mice that
cleared \emph{C. difficile}, but shifted after 5-days post-challenge so
that it was increased in mice that had prolonged colonization. In the
context of CDIs, some studies suggest a protective role (45, 46), while
others suggest detrimental (47--49). Because the relative abundance of
\emph{Akkermansia} was dynamic in our study so it is unclear whether
\emph{Akkermansia} helps with clearance of \emph{C. difficile} or allows
it to persist. A better understanding how \emph{C. difficile} interacts
with the mucosal microbiota may lead to insights into CDIs, asymptomatic
\emph{C. difficile} colonization, and resistance.

Despite identifying an altered compositional profile that included
higher relative abundance of the \emph{C. difficile} sequence in the
mucosal tissues of mice treated with 5-day PEG compared to the
clindamycin group, we did not see a difference in histopathology scores
between the groups. One reason there was no difference could be the
\emph{C. difficile} strain used, \emph{C. difficile} 630 results in mild
histopathology summary scores in mice compared to VPI 10463 despite both
strains producing toxin in mice (50). Part of our hypothesis for why
there could have been increased histopathology scores in PEG-treated
mice was because PEG was previously shown to disrupt the mucus layer in
mice. However, recent studies demonstrated that broad spectrum
antibiotics can also disrupt the host mucosal barrier in mice (51, 52).
Future research is needed to tease out the interplay between medications
that influence the mucus layer and different strains of \emph{C.
difficile} in the context of CDIs.

It is more difficult what are findings mean in the context of \emph{C.
difficile} colonization resistance mean for human patients. Considering
that most hospitals recommend not performing \emph{C. difficile} testing
on patients taking laxatives and laxatives are used when administering
fecal microbiota transplants via colonoscopy to patients with recurrent
CDIs. further studies are needed to evaluate if laxatives impact human
microbiota colonization resistance. Further studies are needed to
understand the impact of osmotic laxatives on \emph{C. difficile}
colonization resistance and clearance in human patients. + What's known
regarding laxatives and susceptibility to CDIs + Clinical trial of PEG,
results never posted (53)

Relevance to human FMTs? Unclear what the best administration route is
because there have been no studies designed to evaluate the best
administration route for FMTs. However, results from the FMT National
Registry where 85\% of FMTs were delivered by colonoscopy demonstrate
FMTs are highly effective treatment for recurrent CDIs with 90\%
achieving resolution in the 1 month follow-up window (54).

We have demonstrated that osmotic laxative treatment alone has a
substantial impact on the microbiota and rendered mice susceptible to
prolonged \emph{C. difficile} colonization in contrast to
clindamycin-treated mice. The duration and timing of the laxative
treatment impacted the duration of \emph{C. difficile} colonization,
with only 5-day PEG and post-challenge 1-day PEG treatments prolonging
colonization compared to clindamycin treated mice. Further studies are
warranted to ascertain whether laxatives have a similar impact on
\emph{C. difficile} colonization resistance on the human microbiota.

\hypertarget{acknowledgements}{%
\subsection{Acknowledgements}\label{acknowledgements}}

We thank members of the Schloss lab for feedback on planning the
experiments and data presentation. We thank Andrew Henry for help with
media preparation and bacterial culture and the Microbiology and
Immunology department's postdoc association writing group members for
their feedback on manuscript drafts. We also thank the Unit for
Laboratory Animal Medicine at the University of Michigan for maintaining
our mouse colony and providing the institutional support for our mouse
experiments. Finally, we thank Kwi Kim, Austin Campbell, and Kimberly
Vendrov for their help in maintaining the Schloss lab's anaerobic
chamber. This work was supported by the National Institutes of Health
(U01AI124255). ST was supported by the Michigan Institute for Clincial
and Health Research Postdoctoral Translation Scholars Program
(UL1TR002240 from the National Center for Advancing Translational
Sciences).

\hypertarget{materials-and-methods}{%
\subsection{Materials and Methods}\label{materials-and-methods}}

\textbf{Animals.} All experiments were approved by the University of
Michigan Animal Care and Use Committee IACUC (protocol numbers
PRO00006983 and PRO00008975). All mice were C57Bl/6 and part of the
Schloss lab colony which was established in 2010 with mice donated from
Vincent Young's lab colony (established with mice purchased from The
Jackson Laboratory in 2002). We used 7-19 week old female mice for all
experiments, which allowed us to break up littermates prior to starting
an experiment and distribute them as evenly as possible across treatment
groups in order to minimize microbiota differences between experiment
groups prior to starting treatments with medications. During the
experiment, mice were housed at a density of 2-3 mice per cage, with the
majority of cages limited to two mice.

\textbf{Drug treatments.} For PEG treament groups, fifteen percent PEG
3350 (Miralax) was administered in the drinking water for either 5 or
1-day periods depending on the experiment. PEG solution was prepared
fresh every 2 days in distilled water and administered to the mice in
water bottles. Clindamycin treatment groups received distilled water in
water bottles during the PEG-treatment periods, with the water being
changed at the same frequency. For clindamycin treatment, groups of mice
received 10 mg/kg clindamycin (Sigma-Aldrich) via intraperitoneal
injection. All PEG treatment groups received a sham intraperitoneal
injection containing filter sterilized saline.

\textbf{\emph{C. difficile} challenge model.} Mice were challenged with
25 microliters off \emph{C. difficile} 630 spores at
10\textsuperscript{5} concentration, except for 1 experiment (Fig. 5A)
where the concentration was 10\textsuperscript{3}. All mock challenged
mice received 25 ul vehicle solution (Ultrapure water). A Dymax stepper
pipette was used to administer the same challenge dose to mice via oral
gavage. Mice were weighed daily throughout the experiment and stool was
collected for quantifying \emph{C. difficile} CFU and 16S rRNA gene
sequencing. There were two groups of mice that received either a PBS or
fecal microbiota transplant (FMT) gavage post-PEG treatment. The fecal
microbiota transplant was prepared with stool samples collected from the
mice in the experiment prior to the start of any treatments. The stool
samples were transferred to an anaerobic chamber and diluted 1:10 in
reduced PBS and glycerol was added to make a 15\% glycerol solution. The
solution was then aliquoted into tubes and stored at -80°C until the day
of the gavage. An aliquot of both the FMT and PBS solutions were also
set aside in the -80°C for 16S rRNA gene sequencing. The day of the
gavage, aliquots were thawed and centrifuged at 7500 RPM for 1 minute.
The supernatant was then transferred to a separate tube to prevent the
gavage needle from clogging with debris during gavage. The PBS solution
that was administered to the other group was also 15\% glycerol. Each
mouse was administered 100 microliters of either the FMT or PBS solution
via gavage. When we refer to mice that cleared \emph{C. difficile}, we
mean that no \emph{C. difficile} was detected in the first serial
dilution (limit of detection: 100 CFU). In some experiments, we
collected tissues for 16SrRNA gene sequencing, histopathology, or both.
For 16S rRNA gene sequencing, we collected small snips of cecum,
proximal colon, and distal colon tissues in microcentrifuge tubes, snap
froze in liquid nitrogen, and stored at -80°C. For histopathology, cecum
and colon tissues were placed into separate casettes, fixed, and then
submitted to McClinchey Histology Labs (Stockbridge, MI) for processing,
embedding, and hematoxylin and eosin (H\&E) staining.

\textbf{\emph{C. difficile} quantification.} Stool samples from mice
were transferred to an anaerobic chamber and serially diluted in reduced
PBS. Serial dilutions were plated onto
taurocholate-cycloserine-cefoxitin-fructose agar (TCCFA) plates plates
and counted after 24 hours of incubation at 37°C. Stool samples
collected from the mice on day 0 post-challenge were also plated onto
TCCFA plates to ensure mice were not already colonized with \emph{C.
difficile} prior to challenge.

\textbf{16S rRNA gene sequencing.} Stool samples that were stored in the
-80°C were placed into 96-well plates for DNA extractions and library
preparation. DNA was extracted using the DNeasy Powersoil HTP 96 kit
(Qiagen) and an EpMotion 5075 automated pipetting system (Eppendorf).
For library preparation, each plate had a mock community control
(ZymoBIOMICS microbial community DNA standards) and a negative control
(water). The V4 region of the 16S rRNA gene was amplified with the
AccuPrime Pfx DNA polymerase (Thermo Fisher Scientific) using custom
barcoded primers, as previously described (55). The PCR amplicons were
normalized (SequalPrep normalizatin plate kit from Thermo Fisher
Scientific), pooled and quantified (KAPA library quantification kit from
KAPA Biosystems), and sequenced with the MiSeq system (Illumina).

\textbf{16S rRNA gene sequence analysis.} All sequences were processed
with mothur (v. 1.43) using a previously published protocol (55, 56).
Paired sequencing reads were combined and aligned with the SILVA (v.
132) reference database (57) and taxonomy was assigned with a modified
version (v. 16) of the Ribosomal Database Project (v. 11.5) (58). The
error rate for are sequencing data was 0.0559\% based on the 17 mock
communities we ran with the samples. Samples were rarefied to 1,000
sequences, 1,000 times for alpha and beta diversity analyses in order to
account for uneven sequencing across samples. All but 3 out of 17 water
controls had less than 1000 sequences. PCoAs were generated based on
Bray-Curtis Index distance matrices. Permutational multivariate analysis
of variance (PERMANOVA) tests were performed on mothur-generated
Bray-Curtis distance matrices with the adonis function from the vegan R
package (59).

\textbf{Histopathology.} H\&E stained sections of cecum and colon
tissues collected at either 0, 4, or 6 days post-challenge were coded to
be scored in a blinded manner by a board-certified veterinary
pathologist (ILB). Slides were evaluated using a scoring system
developed for mouse models of \emph{C. difficile} infection (50). Each
slide was evaluated for edema, cellular infiltration, and inflammation
and given a score ranging from 0-4. The summary score was calculated by
combining the scores from the 3 categories and ranged from 0-12.

\textbf{Classification model training and evaluation.} We used the
mikropml package to train and evaluate models to predict \emph{C.
difficile} colonization status 10 days post-challenge where mice were
categorized as either cleared or colonized (60, 61). We removed the
\emph{C. difficile} genus relative abundance data prior to training the
model. Input community relative abundance data at the genus level from 5
days post-challenge was used to generate random forest, logistic
regression, and support vector machine classification models to predict
\emph{C. difficile} colonization status 10 days post-challenge. To
accommodate the small number of samples in our data set we used 50\%
training and 50\% testing splits with repeated 2-fold cross-validation
of the training data for hyperparamter tuning. Permutation importance
was performed as described previously (62) using mikropml (60, 61) with
the random forest model because it had the highest AUROC value.

\textbf{Statistical analysis.} R (v. 4.0.2) and the tidyverse package
(v. 1.3.0) were used for statistical analysis (63, 64). Kruskal-Wallis
tests with Bejamini-Hochberg correction for testing multiple time points
were used to analyze differences in \emph{C. difficile} CFU, mouse
weight change, and alpha diversity between treatment groups. Paired
Wilcoxon rank signed rank tests were used to identify genera impacted by
treatments on matched pairs of samples from 2 time points. Bacterial
relative abundances that varied between treatment groups at the genus
level were identified with the Kruskal-Wallis test with
Benjamini-Hochberg correction for testing all identified OTUs, followed
by pairwise Wilcoxon comparisons with Benjamini-Hocherg correction.

\textbf{Code availability.} Code for data analysis and generating this
paper with accompanying figures is available at
\url{https://github.com/SchlossLab/Tomkovich_PEG3350_XXXX_2021}.

\textbf{Data availability.} The 16S rRNA sequencing data have been
deposited in the National Center for Biotechnology Information Sequence
Read Archive (BioProject Accession no. PRJNA727293).

\newpage

\hypertarget{references}{%
\subsection{References}\label{references}}

\hypertarget{refs}{}
\leavevmode\hypertarget{ref-Britton2014}{}%
1. \textbf{Britton RA}, \textbf{Young VB}. 2014. Role of the intestinal
microbiota in resistance to colonization by clostridium difficile.
Gastroenterology \textbf{146}:1547--1553.
doi:\href{https://doi.org/10.1053/j.gastro.2014.01.059}{10.1053/j.gastro.2014.01.059}.

\leavevmode\hypertarget{ref-Maier2018}{}%
2. \textbf{Maier L}, \textbf{Pruteanu M}, \textbf{Kuhn M},
\textbf{Zeller G}, \textbf{Telzerow A}, \textbf{Anderson EE},
\textbf{Brochado AR}, \textbf{Fernandez KC}, \textbf{Dose H},
\textbf{Mori H}, \textbf{Patil KR}, \textbf{Bork P}, \textbf{Typas A}.
2018. Extensive impact of non-antibiotic drugs on human gut bacteria.
Nature \textbf{555}:623--628.
doi:\href{https://doi.org/10.1038/nature25979}{10.1038/nature25979}.

\leavevmode\hypertarget{ref-LeBastard2017}{}%
3. \textbf{Bastard QL}, \textbf{Al-Ghalith GA}, \textbf{Grégoire M},
\textbf{Chapelet G}, \textbf{Javaudin F}, \textbf{Dailly E},
\textbf{Batard E}, \textbf{Knights D}, \textbf{Montassier E}. 2017.
Systematic review: Human gut dysbiosis induced by non-antibiotic
prescription medications. Alimentary Pharmacology \& Therapeutics
\textbf{47}:332--345.
doi:\href{https://doi.org/10.1111/apt.14451}{10.1111/apt.14451}.

\leavevmode\hypertarget{ref-VichVila2020}{}%
4. \textbf{Vila AV}, \textbf{Collij V}, \textbf{Sanna S}, \textbf{Sinha
T}, \textbf{Imhann F}, \textbf{Bourgonje AR}, \textbf{Mujagic Z},
\textbf{Jonkers DMAE}, \textbf{Masclee AAM}, \textbf{Fu J},
\textbf{Kurilshikov A}, \textbf{Wijmenga C}, \textbf{Zhernakova A},
\textbf{Weersma RK}. 2020. Impact of commonly used drugs on the
composition and metabolic function of the gut microbiota. Nature
Communications \textbf{11}.
doi:\href{https://doi.org/10.1038/s41467-019-14177-z}{10.1038/s41467-019-14177-z}.

\leavevmode\hypertarget{ref-Oh2018}{}%
5. \textbf{Oh J}, \textbf{Makar M}, \textbf{Fusco C}, \textbf{McCaffrey
R}, \textbf{Rao K}, \textbf{Ryan EE}, \textbf{Washer L}, \textbf{West
LR}, \textbf{Young VB}, \textbf{Guttag J}, \textbf{Hooper DC},
\textbf{Shenoy ES}, \textbf{Wiens J}. 2018. A generalizable, data-driven
approach to predict daily risk of clostridium difficile infection at two
large academic health centers. Infection Control \& Hospital
Epidemiology \textbf{39}:425--433.
doi:\href{https://doi.org/10.1017/ice.2018.16}{10.1017/ice.2018.16}.

\leavevmode\hypertarget{ref-Mora2012}{}%
6. \textbf{Mora AL}, \textbf{Salazar M}, \textbf{Pablo-Caeiro J},
\textbf{Frost CP}, \textbf{Yadav Y}, \textbf{DuPont HL}, \textbf{Garey
KW}. 2012. Moderate to high use of opioid analgesics are associated with
an increased risk of clostridium difficile infection. The American
Journal of the Medical Sciences \textbf{343}:277--280.
doi:\href{https://doi.org/10.1097/maj.0b013e31822f42eb}{10.1097/maj.0b013e31822f42eb}.

\leavevmode\hypertarget{ref-Nehra2018}{}%
7. \textbf{Nehra AK}, \textbf{Alexander JA}, \textbf{Loftus CG},
\textbf{Nehra V}. 2018. Proton pump inhibitors: Review of emerging
concerns. Mayo Clinic Proceedings \textbf{93}:240--246.
doi:\href{https://doi.org/10.1016/j.mayocp.2017.10.022}{10.1016/j.mayocp.2017.10.022}.

\leavevmode\hypertarget{ref-Krishna2013}{}%
8. \textbf{Krishna SG}, \textbf{Zhao W}, \textbf{Apewokin SK},
\textbf{Krishna K}, \textbf{Chepyala P}, \textbf{Anaissie EJ}. 2013.
Risk factors, preemptive therapy, and antiperistaltic agents
forClostridium difficileinfection in cancer patients. Transplant
Infectious Disease n/a--n/a.
doi:\href{https://doi.org/10.1111/tid.12112}{10.1111/tid.12112}.

\leavevmode\hypertarget{ref-Tomkovich2019}{}%
9. \textbf{Tomkovich S}, \textbf{Lesniak NA}, \textbf{Li Y},
\textbf{Bishop L}, \textbf{Fitzgerald MJ}, \textbf{Schloss PD}. 2019.
The proton pump inhibitor omeprazole does not promote
\emph{Clostridioides difficile} colonization in a murine model. mSphere
\textbf{4}.
doi:\href{https://doi.org/10.1128/msphere.00693-19}{10.1128/msphere.00693-19}.

\leavevmode\hypertarget{ref-Vandeputte2015}{}%
10. \textbf{Vandeputte D}, \textbf{Falony G}, \textbf{Vieira-Silva S},
\textbf{Tito RY}, \textbf{Joossens M}, \textbf{Raes J}. 2015. Stool
consistency is strongly associated with gut microbiota richness and
composition, enterotypes and bacterial growth rates. Gut
\textbf{65}:57--62.
doi:\href{https://doi.org/10.1136/gutjnl-2015-309618}{10.1136/gutjnl-2015-309618}.

\leavevmode\hypertarget{ref-VujkovicCvijin2020}{}%
11. \textbf{Vujkovic-Cvijin I}, \textbf{Sklar J}, \textbf{Jiang L},
\textbf{Natarajan L}, \textbf{Knight R}, \textbf{Belkaid Y}. 2020. Host
variables confound gut microbiota studies of human disease. Nature
\textbf{587}:448--454.
doi:\href{https://doi.org/10.1038/s41586-020-2881-9}{10.1038/s41586-020-2881-9}.

\leavevmode\hypertarget{ref-Schubert2014}{}%
12. \textbf{Schubert AM}, \textbf{Rogers MAM}, \textbf{Ring C},
\textbf{Mogle J}, \textbf{Petrosino JP}, \textbf{Young VB},
\textbf{Aronoff DM}, \textbf{Schloss PD}. 2014. Microbiome data
distinguish patients with clostridium difficile infection and non-c.
Difficile-associated diarrhea from healthy controls. mBio \textbf{5}.
doi:\href{https://doi.org/10.1128/mbio.01021-14}{10.1128/mbio.01021-14}.

\leavevmode\hypertarget{ref-Nagata2019}{}%
13. \textbf{Nagata N}, \textbf{Tohya M}, \textbf{Fukuda S}, \textbf{Suda
W}, \textbf{Nishijima S}, \textbf{Takeuchi F}, \textbf{Ohsugi M},
\textbf{Tsujimoto T}, \textbf{Nakamura T}, \textbf{Shimomura A},
\textbf{Yanagisawa N}, \textbf{Hisada Y}, \textbf{Watanabe K},
\textbf{Imbe K}, \textbf{Akiyama J}, \textbf{Mizokami M},
\textbf{Miyoshi-Akiyama T}, \textbf{Uemura N}, \textbf{Hattori M}. 2019.
Effects of bowel preparation on the human gut microbiome and metabolome.
Scientific Reports \textbf{9}.
doi:\href{https://doi.org/10.1038/s41598-019-40182-9}{10.1038/s41598-019-40182-9}.

\leavevmode\hypertarget{ref-Kashyap2013}{}%
14. \textbf{Kashyap PC}, \textbf{Marcobal A}, \textbf{Ursell LK},
\textbf{Larauche M}, \textbf{Duboc H}, \textbf{Earle KA},
\textbf{Sonnenburg ED}, \textbf{Ferreyra JA}, \textbf{Higginbottom SK},
\textbf{Million M}, \textbf{Tache Y}, \textbf{Pasricha PJ},
\textbf{Knight R}, \textbf{Farrugia G}, \textbf{Sonnenburg JL}. 2013.
Complex interactions among diet, gastrointestinal transit, and gut
microbiota in humanized mice. Gastroenterology \textbf{144}:967--977.
doi:\href{https://doi.org/10.1053/j.gastro.2013.01.047}{10.1053/j.gastro.2013.01.047}.

\leavevmode\hypertarget{ref-Ferreyra2014}{}%
15. \textbf{Ferreyra JA}, \textbf{Wu KJ}, \textbf{Hryckowian AJ},
\textbf{Bouley DM}, \textbf{Weimer BC}, \textbf{Sonnenburg JL}. 2014.
Gut microbiota-produced succinate promotes c.~difficile infection after
antibiotic treatment or motility disturbance. Cell Host \& Microbe
\textbf{16}:770--777.
doi:\href{https://doi.org/10.1016/j.chom.2014.11.003}{10.1016/j.chom.2014.11.003}.

\leavevmode\hypertarget{ref-Tropini2018}{}%
16. \textbf{Tropini C}, \textbf{Moss EL}, \textbf{Merrill BD},
\textbf{Ng KM}, \textbf{Higginbottom SK}, \textbf{Casavant EP},
\textbf{Gonzalez CG}, \textbf{Fremin B}, \textbf{Bouley DM},
\textbf{Elias JE}, \textbf{Bhatt AS}, \textbf{Huang KC},
\textbf{Sonnenburg JL}. 2018. Transient osmotic perturbation causes
long-term alteration to the gut microbiota. Cell
\textbf{173}:1742--1754.e17.
doi:\href{https://doi.org/10.1016/j.cell.2018.05.008}{10.1016/j.cell.2018.05.008}.

\leavevmode\hypertarget{ref-VanInsberghe2020}{}%
17. \textbf{VanInsberghe D}, \textbf{Elsherbini JA}, \textbf{Varian B},
\textbf{Poutahidis T}, \textbf{Erdman S}, \textbf{Polz MF}. 2020.
Diarrhoeal events can trigger long-term clostridium difficile
colonization with recurrent blooms. Nature Microbiology
\textbf{5}:642--650.
doi:\href{https://doi.org/10.1038/s41564-020-0668-2}{10.1038/s41564-020-0668-2}.

\leavevmode\hypertarget{ref-Olson2013}{}%
18. \textbf{Olson A}, \textbf{Diebel LN}, \textbf{Liberati DM}. 2013.
Effect of host defenses on clostridium difficile toxininduced intestinal
barrier injury. Journal of Trauma and Acute Care Surgery
\textbf{74}:983--990.
doi:\href{https://doi.org/10.1097/ta.0b013e3182858477}{10.1097/ta.0b013e3182858477}.

\leavevmode\hypertarget{ref-Diebel2014}{}%
19. \textbf{Diebel LN}, \textbf{Liberati DM}. 2014. Reinforcement of the
intestinal mucus layer protects against clostridium difficile intestinal
injury in~vitro. Journal of the American College of Surgeons
\textbf{219}:460--468.
doi:\href{https://doi.org/10.1016/j.jamcollsurg.2014.05.005}{10.1016/j.jamcollsurg.2014.05.005}.

\leavevmode\hypertarget{ref-vanNood2013}{}%
20. \textbf{Nood E van}, \textbf{Vrieze A}, \textbf{Nieuwdorp M},
\textbf{Fuentes S}, \textbf{Zoetendal EG}, \textbf{Vos WM de},
\textbf{Visser CE}, \textbf{Kuijper EJ}, \textbf{Bartelsman JFWM},
\textbf{Tijssen JGP}, \textbf{Speelman P}, \textbf{Dijkgraaf MGW},
\textbf{Keller JJ}. 2013. Duodenal infusion of donor feces for
RecurrentClostridium difficile. New England Journal of Medicine
\textbf{368}:407--415.
doi:\href{https://doi.org/10.1056/nejmoa1205037}{10.1056/nejmoa1205037}.

\leavevmode\hypertarget{ref-Razik2017}{}%
21. \textbf{Razik R}, \textbf{Osman M}, \textbf{Lieberman A},
\textbf{Allegretti JR}, \textbf{Kassam Z}. 2017. Faecal microbiota
transplantation for clostridium difficile infection: A multicentre study
of non-responders. Medical Journal of Australia \textbf{207}:159--160.
doi:\href{https://doi.org/10.5694/mja16.01452}{10.5694/mja16.01452}.

\leavevmode\hypertarget{ref-Postigo2012}{}%
22. \textbf{Postigo R}, \textbf{Kim JH}. 2012. Colonoscopic versus
nasogastric fecal transplantation for the treatment of clostridium
difficile infection: A review and pooled analysis. Infection
\textbf{40}:643--648.
doi:\href{https://doi.org/10.1007/s15010-012-0307-9}{10.1007/s15010-012-0307-9}.

\leavevmode\hypertarget{ref-Liacouras1996}{}%
23. \textbf{Liacouras CA}, \textbf{Piccoli DA}. 1996. Whole-bowel
irrigation as an adjunct to the treatment of chronic, relapsing
clostridium difficile colitis. Journal of Clinical Gastroenterology
\textbf{22}:186--189.
doi:\href{https://doi.org/10.1097/00004836-199604000-00007}{10.1097/00004836-199604000-00007}.

\leavevmode\hypertarget{ref-Schubert2015}{}%
24. \textbf{Schubert AM}, \textbf{Sinani H}, \textbf{Schloss PD}. 2015.
Antibiotic-induced alterations of the murine gut microbiota and
subsequent effects on colonization resistance against \emph{Clostridium
difficile}. mBio \textbf{6}.
doi:\href{https://doi.org/10.1128/mbio.00974-15}{10.1128/mbio.00974-15}.

\leavevmode\hypertarget{ref-Jenior2017}{}%
25. \textbf{Jenior ML}, \textbf{Leslie JL}, \textbf{Young VB},
\textbf{Schloss PD}. 2017. \emph{Clostridium difficile} colonizes
alternative nutrient niches during infection across distinct murine gut
microbiomes. mSystems \textbf{2}.
doi:\href{https://doi.org/10.1128/msystems.00063-17}{10.1128/msystems.00063-17}.

\leavevmode\hypertarget{ref-Tomkovich2020}{}%
26. \textbf{Tomkovich S}, \textbf{Stough JMA}, \textbf{Bishop L},
\textbf{Schloss PD}. 2020. The initial gut microbiota and response to
antibiotic perturbation influence clostridioides difficile clearance in
mice. mSphere \textbf{5}.
doi:\href{https://doi.org/10.1128/msphere.00869-20}{10.1128/msphere.00869-20}.

\leavevmode\hypertarget{ref-Nguyen2015}{}%
27. \textbf{Nguyen TLA}, \textbf{Vieira-Silva S}, \textbf{Liston A},
\textbf{Raes J}. 2015. How informative is the mouse for human gut
microbiota research? Disease Models \& Mechanisms \textbf{8}:1--16.
doi:\href{https://doi.org/10.1242/dmm.017400}{10.1242/dmm.017400}.

\leavevmode\hypertarget{ref-Reeves2011}{}%
28. \textbf{Reeves AE}, \textbf{Theriot CM}, \textbf{Bergin IL},
\textbf{Huffnagle GB}, \textbf{Schloss PD}, \textbf{Young VB}. 2011. The
interplay between microbiome dynamics and pathogen dynamics in a murine
model of \emph{Clostridium difficile} infection \textbf{2}:145--158.
doi:\href{https://doi.org/10.4161/gmic.2.3.16333}{10.4161/gmic.2.3.16333}.

\leavevmode\hypertarget{ref-Dieterle2020}{}%
29. \textbf{Dieterle MG}, \textbf{Putler R}, \textbf{Perry DA},
\textbf{Menon A}, \textbf{Abernathy-Close L}, \textbf{Perlman NS},
\textbf{Penkevich A}, \textbf{Standke A}, \textbf{Keidan M},
\textbf{Vendrov KC}, \textbf{Bergin IL}, \textbf{Young VB}, \textbf{Rao
K}. 2020. Systemic inflammatory mediators are effective biomarkers for
predicting adverse outcomes in clostridioides difficile infection. mBio
\textbf{11}.
doi:\href{https://doi.org/10.1128/mbio.00180-20}{10.1128/mbio.00180-20}.

\leavevmode\hypertarget{ref-Jenior2018}{}%
30. \textbf{Jenior ML}, \textbf{Leslie JL}, \textbf{Young VB},
\textbf{Schloss PD}. 2018. \emph{Clostridium difficile} alters the
structure and metabolism of distinct cecal microbiomes during initial
infection to promote sustained colonization. mSphere \textbf{3}.
doi:\href{https://doi.org/10.1128/msphere.00261-18}{10.1128/msphere.00261-18}.

\leavevmode\hypertarget{ref-Iino2007}{}%
31. \textbf{Iino T}, \textbf{Mori K}, \textbf{Tanaka K}, \textbf{Suzuki
K-i}, \textbf{Harayama S}. 2007. Oscillibacter valericigenes gen. Nov.,
sp. Nov., a valerate-producing anaerobic bacterium isolated from the
alimentary canal of a japanese corbicula clam. International Journal of
Systematic and Evolutionary Microbiology \textbf{57}:1840--1845.
doi:\href{https://doi.org/10.1099/ijs.0.64717-0}{10.1099/ijs.0.64717-0}.

\leavevmode\hypertarget{ref-Jump2014}{}%
32. \textbf{Jump RLP}, \textbf{Polinkovsky A}, \textbf{Hurless K},
\textbf{Sitzlar B}, \textbf{Eckart K}, \textbf{Tomas M},
\textbf{Deshpande A}, \textbf{Nerandzic MM}, \textbf{Donskey CJ}. 2014.
Metabolomics analysis identifies intestinal microbiota-derived
biomarkers of colonization resistance in clindamycin-treated mice. PLoS
ONE \textbf{9}:e101267.
doi:\href{https://doi.org/10.1371/journal.pone.0101267}{10.1371/journal.pone.0101267}.

\leavevmode\hypertarget{ref-McDonald2018}{}%
33. \textbf{McDonald JAK}, \textbf{Mullish BH}, \textbf{Pechlivanis A},
\textbf{Liu Z}, \textbf{Brignardello J}, \textbf{Kao D}, \textbf{Holmes
E}, \textbf{Li JV}, \textbf{Clarke TB}, \textbf{Thursz MR},
\textbf{Marchesi JR}. 2018. Inhibiting growth of clostridioides
difficile by restoring valerate, produced by the intestinal microbiota.
Gastroenterology \textbf{155}:1495--1507.e15.
doi:\href{https://doi.org/10.1053/j.gastro.2018.07.014}{10.1053/j.gastro.2018.07.014}.

\leavevmode\hypertarget{ref-Pfeiffer2012}{}%
34. \textbf{Pfeiffer N}, \textbf{Desmarchelier C}, \textbf{Blaut M},
\textbf{Daniel H}, \textbf{Haller D}, \textbf{Clavel T}. 2012.
Acetatifactor muris gen. Nov., sp. Nov., a novel bacterium isolated from
the intestine of an obese mouse. Archives of Microbiology
\textbf{194}:901--907.
doi:\href{https://doi.org/10.1007/s00203-012-0822-1}{10.1007/s00203-012-0822-1}.

\leavevmode\hypertarget{ref-Lawley2012}{}%
35. \textbf{Lawley TD}, \textbf{Clare S}, \textbf{Walker AW},
\textbf{Stares MD}, \textbf{Connor TR}, \textbf{Raisen C},
\textbf{Goulding D}, \textbf{Rad R}, \textbf{Schreiber F},
\textbf{Brandt C}, \textbf{Deakin LJ}, \textbf{Pickard DJ},
\textbf{Duncan SH}, \textbf{Flint HJ}, \textbf{Clark TG},
\textbf{Parkhill J}, \textbf{Dougan G}. 2012. Targeted restoration of
the intestinal microbiota with a simple, defined bacteriotherapy
resolves relapsing \emph{Clostridium difficile} disease in mice. PLoS
Pathogens \textbf{8}:e1002995.
doi:\href{https://doi.org/10.1371/journal.ppat.1002995}{10.1371/journal.ppat.1002995}.

\leavevmode\hypertarget{ref-Frost2021}{}%
36. \textbf{Frost LR}, \textbf{Cheng JKJ}, \textbf{Unnikrishnan M}.
2021. Clostridioides difficile biofilms: A mechanism of persistence in
the gut? PLOS Pathogens \textbf{17}:e1009348.
doi:\href{https://doi.org/10.1371/journal.ppat.1009348}{10.1371/journal.ppat.1009348}.

\leavevmode\hypertarget{ref-Engevik2021}{}%
37. \textbf{Engevik MA}, \textbf{Danhof HA}, \textbf{Auchtung J},
\textbf{Endres BT}, \textbf{Ruan W}, \textbf{Bassères E},
\textbf{Engevik AC}, \textbf{Wu Q}, \textbf{Nicholson M}, \textbf{Luna
RA}, \textbf{Garey KW}, \textbf{Crawford SE}, \textbf{Estes MK},
\textbf{Lux R}, \textbf{Yacyshyn MB}, \textbf{Yacyshyn B},
\textbf{Savidge T}, \textbf{Britton RA}, \textbf{Versalovic J}. 2021.
Fusobacterium nucleatum adheres to clostridioides difficile via the RadD
adhesin to enhance biofilm formation in intestinal mucus.
Gastroenterology \textbf{160}:1301--1314.e8.
doi:\href{https://doi.org/10.1053/j.gastro.2020.11.034}{10.1053/j.gastro.2020.11.034}.

\leavevmode\hypertarget{ref-Engevik2020}{}%
38. \textbf{Engevik MA}, \textbf{Engevik AC}, \textbf{Engevik KA},
\textbf{Auchtung JM}, \textbf{Chang-Graham AL}, \textbf{Ruan W},
\textbf{Luna RA}, \textbf{Hyser JM}, \textbf{Spinler JK},
\textbf{Versalovic J}. 2020. Mucin-degrading microbes release
monosaccharides that chemoattract clostridioides difficile and
facilitate colonization of the human intestinal mucus layer. ACS
Infectious Diseases \textbf{7}:1126--1142.
doi:\href{https://doi.org/10.1021/acsinfecdis.0c00634}{10.1021/acsinfecdis.0c00634}.

\leavevmode\hypertarget{ref-Dejea2018}{}%
39. \textbf{Dejea CM}, \textbf{Fathi P}, \textbf{Craig JM},
\textbf{Boleij A}, \textbf{Taddese R}, \textbf{Geis AL}, \textbf{Wu X},
\textbf{Shields CED}, \textbf{Hechenbleikner EM}, \textbf{Huso DL},
\textbf{Anders RA}, \textbf{Giardiello FM}, \textbf{Wick EC},
\textbf{Wang H}, \textbf{Wu S}, \textbf{Pardoll DM}, \textbf{Housseau
F}, \textbf{Sears CL}. 2018. Patients with familial adenomatous
polyposis harbor colonic biofilms containing tumorigenic bacteria.
Science \textbf{359}:592--597.
doi:\href{https://doi.org/10.1126/science.aah3648}{10.1126/science.aah3648}.

\leavevmode\hypertarget{ref-Semenyuk2015}{}%
40. \textbf{Semenyuk EG}, \textbf{Poroyko VA}, \textbf{Johnston PF},
\textbf{Jones SE}, \textbf{Knight KL}, \textbf{Gerding DN},
\textbf{Driks A}. 2015. Analysis of bacterial communities during
clostridium difficile infection in the mouse. Infection and Immunity
\textbf{83}:4383--4391.
doi:\href{https://doi.org/10.1128/iai.00145-15}{10.1128/iai.00145-15}.

\leavevmode\hypertarget{ref-Deng2018}{}%
41. \textbf{Deng H}, \textbf{Yang S}, \textbf{Zhang Y}, \textbf{Qian K},
\textbf{Zhang Z}, \textbf{Liu Y}, \textbf{Wang Y}, \textbf{Bai Y},
\textbf{Fan H}, \textbf{Zhao X}, \textbf{Zhi F}. 2018. Bacteroides
fragilis prevents clostridium difficile infection in a mouse model by
restoring gut barrier and microbiome regulation. Frontiers in
Microbiology \textbf{9}.
doi:\href{https://doi.org/10.3389/fmicb.2018.02976}{10.3389/fmicb.2018.02976}.

\leavevmode\hypertarget{ref-Normington2021}{}%
42. \textbf{Normington C}, \textbf{Moura IB}, \textbf{Bryant JA},
\textbf{Ewin DJ}, \textbf{Clark EV}, \textbf{Kettle MJ}, \textbf{Harris
HC}, \textbf{Spittal W}, \textbf{Davis G}, \textbf{Henn MR},
\textbf{Ford CB}, \textbf{Wilcox MH}, \textbf{Buckley AM}. 2021.
Biofilms harbour clostridioides difficile, serving as a reservoir for
recurrent infection. npj Biofilms and Microbiomes \textbf{7}.
doi:\href{https://doi.org/10.1038/s41522-021-00184-w}{10.1038/s41522-021-00184-w}.

\leavevmode\hypertarget{ref-Cirstea2018}{}%
43. \textbf{Cirstea M}, \textbf{Radisavljevic N}, \textbf{Finlay BB}.
2018. Good bug, bad bug: Breaking through microbial stereotypes. Cell
Host \& Microbe \textbf{23}:10--13.
doi:\href{https://doi.org/10.1016/j.chom.2017.12.008}{10.1016/j.chom.2017.12.008}.

\leavevmode\hypertarget{ref-BorgesCanha2015}{}%
44. \textbf{Canha MB}. 2015. Role of colonic microbiota in colorectal
carcinogenesis: A systematic review. Revista Española de Enfermedades
Digestivas \textbf{107}.
doi:\href{https://doi.org/10.17235/reed.2015.3830/2015}{10.17235/reed.2015.3830/2015}.

\leavevmode\hypertarget{ref-Pereira2020}{}%
45. \textbf{Pereira FC}, \textbf{Wasmund K}, \textbf{Cobankovic I},
\textbf{Jehmlich N}, \textbf{Herbold CW}, \textbf{Lee KS},
\textbf{Sziranyi B}, \textbf{Vesely C}, \textbf{Decker T},
\textbf{Stocker R}, \textbf{Warth B}, \textbf{Bergen M von},
\textbf{Wagner M}, \textbf{Berry D}. 2020. Rational design of a
microbial consortium of mucosal sugar utilizers reduces clostridiodes
difficile colonization. Nature Communications \textbf{11}.
doi:\href{https://doi.org/10.1038/s41467-020-18928-1}{10.1038/s41467-020-18928-1}.

\leavevmode\hypertarget{ref-Rodriguez2016}{}%
46. \textbf{Rodriguez C}, \textbf{Taminiau B}, \textbf{Korsak N},
\textbf{Avesani V}, \textbf{Broeck JV}, \textbf{Brach P}, \textbf{Delmée
M}, \textbf{Daube G}. 2016. Longitudinal survey of clostridium difficile
presence and gut microbiota composition in a belgian nursing home. BMC
Microbiology \textbf{16}.
doi:\href{https://doi.org/10.1186/s12866-016-0848-7}{10.1186/s12866-016-0848-7}.

\leavevmode\hypertarget{ref-Sangster2016}{}%
47. \textbf{Sangster W}, \textbf{Hegarty JP}, \textbf{Schieffer KM},
\textbf{Wright JR}, \textbf{Hackman J}, \textbf{Toole DR},
\textbf{Lamendella R}, \textbf{Stewart DB}. 2016. Bacterial and fungal
microbiota changes distinguish c. Difficile infection from other forms
of diarrhea: Results of a prospective inpatient study. Frontiers in
Microbiology \textbf{7}.
doi:\href{https://doi.org/10.3389/fmicb.2016.00789}{10.3389/fmicb.2016.00789}.

\leavevmode\hypertarget{ref-Vakili2020a}{}%
48. \textbf{Vakili B}, \textbf{Fateh A}, \textbf{Aghdaei HA},
\textbf{Sotoodehnejadnematalahi F}, \textbf{Siadat SD}. 2020. Intestinal
microbiota in elderly inpatients with clostridioides difficile
infection. Infection and Drug Resistance \textbf{Volume 13}:2723--2731.
doi:\href{https://doi.org/10.2147/idr.s262019}{10.2147/idr.s262019}.

\leavevmode\hypertarget{ref-Vakili2020b}{}%
49. \textbf{Vakili B}, \textbf{Fateh A}, \textbf{Aghdaei HA},
\textbf{Sotoodehnejadnematalahi F}, \textbf{Siadat SD}. 2020.
Characterization of gut microbiota in hospitalized patients with
clostridioides difficile infection. Current Microbiology
\textbf{77}:1673--1680.
doi:\href{https://doi.org/10.1007/s00284-020-01980-x}{10.1007/s00284-020-01980-x}.

\leavevmode\hypertarget{ref-Theriot2011}{}%
50. \textbf{Theriot CM}, \textbf{Koumpouras CC}, \textbf{Carlson PE},
\textbf{Bergin II}, \textbf{Aronoff DM}, \textbf{Young VB}. 2011.
Cefoperazone-treated mice as an experimental platform to assess
differential virulence ofClostridium difficilestrains. Gut Microbes
\textbf{2}:326--334.
doi:\href{https://doi.org/10.4161/gmic.19142}{10.4161/gmic.19142}.

\leavevmode\hypertarget{ref-Kester2020}{}%
51. \textbf{Kester JC}, \textbf{Brubaker DK}, \textbf{Velazquez J},
\textbf{Wright C}, \textbf{Lauffenburger DA}, \textbf{Griffith LG}.
2020. Clostridioides difficile-associated antibiotics alter human
mucosal barrier functions by microbiome-independent mechanisms.
Antimicrobial Agents and Chemotherapy \textbf{64}.
doi:\href{https://doi.org/10.1128/aac.01404-19}{10.1128/aac.01404-19}.

\leavevmode\hypertarget{ref-Bergstrom2020}{}%
52. \textbf{Bergstrom K}, \textbf{Shan X}, \textbf{Casero D},
\textbf{Batushansky A}, \textbf{Lagishetty V}, \textbf{Jacobs JP},
\textbf{Hoover C}, \textbf{Kondo Y}, \textbf{Shao B}, \textbf{Gao L},
\textbf{Zandberg W}, \textbf{Noyovitz B}, \textbf{McDaniel JM},
\textbf{Gibson DL}, \textbf{Pakpour S}, \textbf{Kazemian N},
\textbf{McGee S}, \textbf{Houchen CW}, \textbf{Rao CV}, \textbf{Griffin
TM}, \textbf{Sonnenburg JL}, \textbf{McEver RP}, \textbf{Braun J},
\textbf{Xia L}. 2020. Proximal colonderived o-glycosylated mucus
encapsulates and modulates the microbiota. Science
\textbf{370}:467--472.
doi:\href{https://doi.org/10.1126/science.aay7367}{10.1126/science.aay7367}.

\leavevmode\hypertarget{ref-Dieterle2018}{}%
53. \textbf{Dieterle MG}, \textbf{Rao K}, \textbf{Young VB}. 2018. Novel
therapies and preventative strategies for primary and
recurrentClostridium difficileinfections. Annals of the New York Academy
of Sciences \textbf{1435}:110--138.
doi:\href{https://doi.org/10.1111/nyas.13958}{10.1111/nyas.13958}.

\leavevmode\hypertarget{ref-Kelly2021}{}%
54. \textbf{Kelly CR}, \textbf{Yen EF}, \textbf{Grinspan AM},
\textbf{Kahn SA}, \textbf{Atreja A}, \textbf{Lewis JD}, \textbf{Moore
TA}, \textbf{Rubin DT}, \textbf{Kim AM}, \textbf{Serra S},
\textbf{Nersesova Y}, \textbf{Fredell L}, \textbf{Hunsicker D},
\textbf{McDonald D}, \textbf{Knight R}, \textbf{Allegretti JR},
\textbf{Pekow J}, \textbf{Absah I}, \textbf{Hsu R}, \textbf{Vincent J},
\textbf{Khanna S}, \textbf{Tangen L}, \textbf{Crawford CV},
\textbf{Mattar MC}, \textbf{Chen LA}, \textbf{Fischer M},
\textbf{Arsenescu RI}, \textbf{Feuerstadt P}, \textbf{Goldstein J},
\textbf{Kerman D}, \textbf{Ehrlich AC}, \textbf{Wu GD}, \textbf{Laine
L}. 2021. Fecal microbiota transplantation is highly effective in
real-world practice: Initial results from the FMT national registry.
Gastroenterology \textbf{160}:183--192.e3.
doi:\href{https://doi.org/10.1053/j.gastro.2020.09.038}{10.1053/j.gastro.2020.09.038}.

\leavevmode\hypertarget{ref-Kozich2013}{}%
55. \textbf{Kozich JJ}, \textbf{Westcott SL}, \textbf{Baxter NT},
\textbf{Highlander SK}, \textbf{Schloss PD}. 2013. Development of a
dual-index sequencing strategy and curation pipeline for analyzing
amplicon sequence data on the MiSeq illumina sequencing platform.
Applied and Environmental Microbiology \textbf{79}:5112--5120.
doi:\href{https://doi.org/10.1128/aem.01043-13}{10.1128/aem.01043-13}.

\leavevmode\hypertarget{ref-Schloss2009}{}%
56. \textbf{Schloss PD}, \textbf{Westcott SL}, \textbf{Ryabin T},
\textbf{Hall JR}, \textbf{Hartmann M}, \textbf{Hollister EB},
\textbf{Lesniewski RA}, \textbf{Oakley BB}, \textbf{Parks DH},
\textbf{Robinson CJ}, \textbf{Sahl JW}, \textbf{Stres B},
\textbf{Thallinger GG}, \textbf{Horn DJV}, \textbf{Weber CF}. 2009.
Introducing mothur: Open-source, platform-independent,
community-supported software for describing and comparing microbial
communities. Applied and Environmental Microbiology
\textbf{75}:7537--7541.
doi:\href{https://doi.org/10.1128/aem.01541-09}{10.1128/aem.01541-09}.

\leavevmode\hypertarget{ref-Quast2012}{}%
57. \textbf{Quast C}, \textbf{Pruesse E}, \textbf{Yilmaz P},
\textbf{Gerken J}, \textbf{Schweer T}, \textbf{Yarza P}, \textbf{Peplies
J}, \textbf{Glöckner FO}. 2012. The SILVA ribosomal RNA gene database
project: Improved data processing and web-based tools. Nucleic Acids
Research \textbf{41}:D590--D596.
doi:\href{https://doi.org/10.1093/nar/gks1219}{10.1093/nar/gks1219}.

\leavevmode\hypertarget{ref-Cole2013}{}%
58. \textbf{Cole JR}, \textbf{Wang Q}, \textbf{Fish JA}, \textbf{Chai
B}, \textbf{McGarrell DM}, \textbf{Sun Y}, \textbf{Brown CT},
\textbf{Porras-Alfaro A}, \textbf{Kuske CR}, \textbf{Tiedje JM}. 2013.
Ribosomal database project: Data and tools for high throughput rRNA
analysis. Nucleic Acids Research \textbf{42}:D633--D642.
doi:\href{https://doi.org/10.1093/nar/gkt1244}{10.1093/nar/gkt1244}.

\leavevmode\hypertarget{ref-Vegan2018}{}%
59. \textbf{Oksanen J}, \textbf{Blanchet FG}, \textbf{Friendly M},
\textbf{Kindt R}, \textbf{Legendre P}, \textbf{McGlinn D},
\textbf{Minchin PR}, \textbf{O'Hara RB}, \textbf{Simpson GL},
\textbf{Solymos P}, \textbf{Stevens MHH}, \textbf{Szoecs E},
\textbf{Wagner H}. 2018. Vegan: Community ecology package.

\leavevmode\hypertarget{ref-mikropml}{}%
60. \textbf{Topçuoğlu B}, \textbf{Lapp Z}, \textbf{Sovacool KL},
\textbf{Snitkin E}, \textbf{Wiens J}, \textbf{Schloss PD}. 2020.
mikRopML: User-friendly r package for robust machine learning pipelines.

\leavevmode\hypertarget{ref-Topcuoglu2021}{}%
61. \textbf{Topçuoğlu B}, \textbf{Lapp Z}, \textbf{Sovacool K},
\textbf{Snitkin E}, \textbf{Wiens J}, \textbf{Schloss P}. 2021.
Mikropml: User-friendly r package for supervised machine learning
pipelines. Journal of Open Source Software \textbf{6}:3073.
doi:\href{https://doi.org/10.21105/joss.03073}{10.21105/joss.03073}.

\leavevmode\hypertarget{ref-Topcuoglu2020}{}%
62. \textbf{Topçuoğlu BD}, \textbf{Lesniak NA}, \textbf{Ruffin MT},
\textbf{Wiens J}, \textbf{Schloss PD}. 2020. A framework for effective
application of machine learning to microbiome-based classification
problems. mBio \textbf{11}.
doi:\href{https://doi.org/10.1128/mbio.00434-20}{10.1128/mbio.00434-20}.

\leavevmode\hypertarget{ref-r_citation_2020}{}%
63. \textbf{R Core Team}. 2020. R: A language and environment for
statistical computing. R Foundation for Statistical Computing, Vienna,
Austria.

\leavevmode\hypertarget{ref-Tidyverse2019}{}%
64. \textbf{Wickham H}, \textbf{Averick M}, \textbf{Bryan J},
\textbf{Chang W}, \textbf{McGowan LD}, \textbf{François R},
\textbf{Grolemund G}, \textbf{Hayes A}, \textbf{Henry L}, \textbf{Hester
J}, \textbf{Kuhn M}, \textbf{Pedersen TL}, \textbf{Miller E},
\textbf{Bache SM}, \textbf{Müller K}, \textbf{Ooms J}, \textbf{Robinson
D}, \textbf{Seidel DP}, \textbf{Spinu V}, \textbf{Takahashi K},
\textbf{Vaughan D}, \textbf{Wilke C}, \textbf{Woo K}, \textbf{Yutani H}.
2019. Welcome to the tidyverse. Journal of Open Source Software
\textbf{4}:1686.
doi:\href{https://doi.org/10.21105/joss.01686}{10.21105/joss.01686}.

\newpage

\includegraphics{figure_1.pdf}

\textbf{Figure 1. 5-day PEG treatment prolongs susceptibility and mice
become persistently colonized with \emph{C. difficile}.} A. Setup of the
experimental time line for experiments with 5-day PEG treated mice
consisting of 4 treatment groups. 1. Clindamycin was administered at 10
mg/kg by intraperitoneal injection. 2. 15\% PEG 3350 was administered in
the drinking water for five days. 3. 5-day PEG plus clindamycin
treatment. 4. 5-day PEG plus 10-day recovery treatment. All treatment
groups were then challenged with 10\textsuperscript{5} \emph{C.
difficile} 630 spores. A subset of mice were euthanized on either 4 or 6
days post-challenge and tissues were collected for histopathology
analysis, the remaining mice were followed through 20 or 30 days
post-challenge. B. Weight change from baseline weight in groups after
treatment with PEG and/or clindamycin, followed by \emph{C. difficile}
challenge. C. \emph{C. difficile} CFU/gram stool measured over time via
serial dilutions(N = 10-59 mice per time point). The black line
represents the limit of detection for the first serial dilution. CFU
quantification data was not available for each mouse due to stool
sampling difficulties (particularly the day the mice came off of the PEG
treatment) or early deaths. For B-C, lines represent the median for each
treatment group and circles represent samples from individual mice.
Asterisks indicate time points where the weight change or CFU/g was
significantly different between groups by the Kruskal-Wallis test with
Benjamini-Hochberg correction for testing multiple time points. The data
presented are from a total of 5 separate experiments. \newpage

\includegraphics{figure_2.pdf} \textbf{Figure 2. 5-day PEG treatment
disrupts the stool microbiota for a longer amount of time compared to
clindamycin-treated mice.} A. Principal Coordinate analysis (PCoA) of
Bray-Curtis distances from stool samples collected throughout the
experiment. Each circle represents a sample from an individual mouse and
the transparency of the symbol corresponds to the day post-challenge. B.
Bray-Curtis distances of stool samples collected on either day 10 or 30
post-challenge relative to the baseline sample collected for each mouse
(before any drug treatments were administered). C. Shannon diversity in
stool communities over time. The line indicates the median value for
each treatment group. For B-C, the symboles represent samples from
individual mice and the line indicates the median value for each
treatment group. D. 14 of the 33 genera affected by PEG treatment (Data
Set S1, sheet X). The symbols represent the median relative abundance
for a treatment group at either baseline (open circle) or 1-day post
treatment (closed circle). Relative abundance data from paired baseline
and 1-day post treatment stool sampes from the 5-day PEG and 5-day PEG
plus 10-day recovery groups were analyzed by paired Wilcoxan signed-rank
test with Benjamini-Hochberg correction for testing all identified
genera. The clindamycin and 5-day PEG plus clindamycin treatment groups
are shown on the plot for comparison. E. 6 of the 24 genera that were
significantly different between the treatment groups over multiple time
points, the 5-day PEG plus clindamycin treatment group was only followed
through 6-days post-challenge. Differences between treatment groups were
identified by Kruskal-Wallis test with Benjamini-Hochberg correction for
testing all identified genera. The gray vertical line (D) and horizontal
vertical lines (E) indicate the limit of detection. \newpage

\includegraphics{figure_3.pdf} \textbf{Figure 3. 5-day PEG treatment
does not result in more severe CDIs, although mucosal microbiota is
altered.} A. Shannon diversity in cecum communities over time. The line
indicates the median value for each treatment group. The colors of the
symbols and lines represent individual and median relative abundance
values for four treatment groups. A similar pattern was observed with
the proximal and distal colon communities (Data Set S1, sheet X-X). B.
PCoA of Bray-Curtis distances from mucosal samples collected throughout
the experiment. Circles, triangles, and squares indicate the cecum,
proximal colon, and distal colon communities, respectively. Transparency
of the symbol corresponds to the day post-challenge that the sample was
collected. C. The median relative abundance of the 4 genera that were
significantly different between the cecum communities of different
treatment groups on day 6 and day 30 post-challenge (Data Set S1, sheet
X). The gray vertical lines indicate the limit of detection. D-E. The
histopathology summary scores from cecum and colon H\&E stained tissue
sections. The summary score is the total score based on evaluation of
edema, cellular infiltration, and inflammation in either the cecum or
colon tissue. Each category is given a score ranging from 0-4, thus the
maximum possible summary score is 12. The tissue for histology was
collected at either 4 (D) or 6 (E) days post-challenge with the
exception that one set of 5-day PEG treated mock-challenged mice were
collected on day 0 post-challenge (first set of open purple circles in
D). Histology data were analyzed with the Kruskal-Wallis test followed
by pairwise Wilcoxon comparisons with Benjamini-Hochberg correction.
\newpage

\includegraphics{figure_4.pdf} \textbf{Figure 4. 1-day PEG treatment
renders mice susceptible to transient \emph{C. difficile} colonization.}
A. Setup of the experimental time line for the 1-day PEG treated mice
consisting of 3 treatment groups. 1. Clindamycin was administered at 10
mg/kg by intraperitoneal injection. 2. 15\% PEG 3350 was administered in
the drinking water for 1 day. 3. 1-day PEG plus 1-day recovery. The
three treatment groups were then challenged with 10\textsuperscript{5}
\emph{C. difficile} 630 spores. B. \emph{C. difficile} CFU/gram stool
measured over time (N = 12-18 mice per time point) by serial dilutions.
The black dashedd horizontal line represents the limit of detection for
the first serial dilution. For B and D, asterisks indicate time points
where there was a significant difference between treatment groups by
Kruskall-Wallis test with Benjamini-Hochberg correction for testing
multiple time points. For B-D, each symbol represents a sample from an
individual mouse and lines indicate the median value for each treatment
group. C. PCoA of Bray-Curtis distances from stool communities collected
over time (day: R\textsuperscript{2} = 0.43; group: R\textsuperscript{2}
= 0.19). Symbol transparency represents the day post-challenge of the
experiment. For C-E, the B on the day legend or days post-challenge X
axis stands for baseline and represents the sample that was collected
prior to any drug treatments. D. Shannon diversity in stool communities
over time. E. Median relative abundances per treatment group for 6 out
of the 14 genera that were affected by treatment, but recovered close to
baseline levels by 7 days post-challenge (Fig. 3E, Data Set S1, Sheet
X). Paired stool sample relative abundance values either baseline and
day 1 or baseline and day 7 were analyzed by paired Wilcoxan signed-rank
test with Benjamini-Hochberg correction for testing all identified
genera. Only genera that were different between baseline and 1-day
post-challenge, but not baseline and 7-days post-challenge. The gray
horizontal lines represents the limit of detection.

\includegraphics{figure_5.pdf} \textbf{Figure 5. 1-day PEG treatment
post \emph{C. difficile} challenge prolongs colonization regardless of
whether an FMT is also administered.} A. Setup of the experimental time
line for experiments with post-challenge PEG treated mice. There were a
total of 4 different treatment groups. All mice were administered 10
mg/kg clindamycin intraperitoneally (IP) 1 day before challenge with
10\textsuperscript{3-5} \emph{C. difficile} 630 spores. 1. Received no
additional treatment (Clindamycin). 2. Immediately after \emph{C.
difficile} challenge, mice received 15\% PEG 3350 in the drinking water
for 1 day. 3-4. 3-days after challenge, mice received 1-day PEG
treatment and then received either 100 microliters a fecal microbiota
transplant (3) or PBS (4) solution by oral gavage. Mice were followed
through 15-30 days post-challenge (only the post-CDI 1-day PEG group was
followed through 30 days post-challenge). B. CFU/g of \emph{C.
difficile} stool measured over time via serial dilutions. The black line
represents the limit of detection for the first serial dilution. C-D.
Shannon diversity (C) and richness (D) in stool communities over time.
B-D. Each symbol represents a stool sample from an individual mouse with
the lines representing the median value for each treatment group.
Asterisks indicate time points with significant differences between
groups by a Kruskall-Wallis test with a Benjamini-Hochberg correction
for testing multiple times points. Colored rectangles indicates the
1-day PEG treatment period for applicable groups. The data presented are
from a total of 3 separate experiments. \newpage

\includegraphics{figure_6.pdf} \textbf{Figure 6. For 1-day PEG treatment
post \emph{C. difficile} challenge mice that also receive an FMT only
some bacterial genera were restored.} A. PCoA of Bray-Curtis distances
from stool samples collected over time as well as the FMT solution that
was administered to one of the treatment groups. Each circle represents
an individual sample, the transparency of the circle corresponds to day
post-challenge. B. Median relative abundances of 2 genera that were
significantly different over multiple time points in mice that were
administered either FMT or PBS solution via gavage C. Median relative
abundances of the top 6 out of 24 genera that were significant over
multiple timepoints, plotted over time (Data Set S1, Sheet X). For B-C,
colored rectangles indicates the 1-day PEG treatment period for
applicable groups. Gray horizontal lines represent the limit of
detection. Differences between treatment groups were identified by
Kruskal-Wallis test with Benjamini-Hochberg correction for testing all
identified genera. For pairwise comparisons of the groups (B), we
performed pairwise Wilcoxon comparisons with Benjamini-Hochberg
correction for testing all combinations of group pairs.

\newpage

\includegraphics{figure_7.pdf} \textbf{Figure 7. Specific microbiota
features associated with prolonged \emph{C. difficile} colonization in
PEG treated mice.} A. Top ten bacteria that contributed to the random
forest model trained on five day post-challenge community relative
abundance data to predict whether mice would still be colonized with
\emph{C. difficile} 10 days post-challenge. The median (point) and
interquartile range (lines) change in AUROC when the bacteria is left
out of the model by permutation feature importance analysis. B. The
median relative abundances of the top ten bacteria that contributed to
the random forest classification model at 5 days post-challenge . Color
indicates whether the mice were still colonized with \emph{C. difficile}
10 days post-challenge and the black horizontal line represents the
median relative abundance for the two categories. Each symbol represents
a stool sample from an individual mouse and the shape of the symbol
indicates whether the PEG-treated mice received a 5-day (Fig. 1-3),
1-day (Fig. 4) or post-challenge PEG (Fig. 5-6) treatment. C. The median
relative abundances of the 5 genera with greater than 1\% median
relative abundance in the stool community over time. For B-C, the gray
horizontal lines represents the limit of detection. \newpage

\includegraphics{figure_8.pdf}

\textbf{Figure 8. Schematic summarizing findings.} The gut microbiota of
our C57Bl/6 mice is resistant to \emph{C. difficle} but treatment with
either the antibiotic, clindamycin, or the osmotic laxative, PEG 3350,
renders the mice susceptible to \emph{C. difficile} colonization.
Recovery of colonization resistance in clindamycin-treated mice is
relatively straightforward and the mice clear \emph{C.difficile} within
10 days post-challenge. However, for mice that received either a 5-day
PEG treatment pre-\emph{C. difficile} challenge or a 1-day PEG treatment
post-challenge recovery of colonization resistance was delayed because
most mice were still colonized with \emph{C. difficile} 10 days
post-challenge. We found increased relative abundances of
\emph{Porphyromonadaceae} and \emph{Lachnospiraceae} were associated
with recovery of colonization resistance, while increased relative
abundances of \emph{Enterobacteriaceae} and \emph{Bacteroides} were
associated with prolonged \emph{C. difficle} colonization. \newpage

\includegraphics{figure_S1.pdf} \textbf{Figure S1. Microbiota dynamics
post-challenge in the 5-day PEG treatment plus 10-day recovery mice.} A.
\emph{C. difficile} CFU/g over time in the stool samples collected from
5-day PEG treated mice that were allowed to recover for 10 days prior to
challenge. Same data presented in Fig. 1C, but the data for the other 3
treatment groups have been removed and each line represents the CFU over
time for an individual mouse. Mouse 10 was found dead 6 days
post-challenge. B. Relative abundances of eight bacterial genera from
day 0 post-challenge onward in each off the 10-day recovery mice. We
analyzed samples from day 0 and day 8 post-challenge, which represented
the the time points where mice were challenged with \emph{C. difficile}
and when the median relative \emph{C. difficile} CFU stabilized for the
group using the paired Wilcoxan signed-rank test, but no genera were
significantly different after Benjamini-Hochberg correction. \newpage

\includegraphics{figure_S2.pdf} \textbf{Figure S2. PEG treatment still
has a large impact on the mucosal microbiota 6 days post-challenge} A.
The relative abundances of the 10 bacterial genera that were
significantly different between treatment groups at 6 days
post-infection in the cecum tissue (the relative abundances of the 10
genera were also significantly different in the proximal and distal
colon tissues, Data Sheet S1, Sheet X). Each symbol represents a tissue
sample from an individual mouse, the black horizontal lines represents
the median relative abundances for each treatment group. B. The relative
abundance of \emph{Peptostreptococacceae} in the three types of tissue
sample communities over time. For A-B, the gray horizontal lines
represent the limit of detection. \newpage

\includegraphics{figure_S3.pdf} \textbf{Figure S3. \emph{C. difficile}
challenge does not enhance the disruptive effect of PEG on the
microbiota.} A, D. PCoAs of the Bray-Curtis distances from the stool (A)
and tissue (D) communities from mock- and \emph{C. difficile}-challenged
treatment groups. Each symbol represents a sample from an individual
mouse with open and closed circles representing mock and \emph{C.
difficile}-challenged mice, respectively. B, E. Median Shannon diversity
in stool (B) and tissue (E) communities collected over time. C, F. The
median relative abundances of genera that were significantly different
between the \emph{C. difficile} challenged treatment groups in either
the stool (Fig. 2E) or cecum tissue (Fig. 3C) communities in the stool
(C) and tissue (F) communities from mock- and \emph{C.
difficile}-challenged mice. For B-F, the dashed and solid lines
represent the median value for mock and \emph{C. difficile}-challenged
mice, respectively. For E-F, tissues from mock-challenged clindamycin
treated mice were only collected 4 days post-challenge so there is no
dashed line for this group. \newpage

\end{document}
